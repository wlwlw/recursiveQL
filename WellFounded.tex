%\begin{align*}, multiline*
Allowing negation in the body of a rule of to have general logic programs may have another problem: As indicated by some research, some programs do not have a satisfactory \emph{total} model. It is thus natural to seek an appropriate \emph{partial} model for such programs.

In this section, \emph{well-founded semantics}, a different approach than stable model semantics in the category of canonical model semantics introduced in [\ref{VanGelder}], is explained and compared to it as well as to other semantics.

\subsection{Preliminaries}
In this subsection, We explain some notions that are used in well-founded semantics.

Given a program $\mathrm{P}$, its \emph{Herbrand universe} is the set of ground terms formed from constants and function symbols in $\mathrm{P}$, and its \emph{Herbrand base} is the set of predicates applied to the ground terms in the Herbrand universe. Note that the program $\mathrm{P}$ may not contain constants or function symbols. In the absence of the former, a constant is added arbitrarily; in the absence of the latter, we then have that both the Herbrand universe and the Herbrand base are countably infinite (otherwise they are finite). An \emph{instantiated rule} of $\mathrm{P}$ is one in which all occurrences of variables are replaced by ground terms in the Herbrand universe. The \emph{Herbrand instantiation} of $\mathrm{P}$ is the set of all instantiated rules.

For a ground atomic formula $p$ (a single predicate applied to ground terms), we say $p$ itself its \emph{positive literal}, and $\neg p$ its \emph{negative literal}; these two literals are \emph{complements} of each other. If $S$ is a set of literals, $\neg \cdot S$ denotes the set obtained by taking the complement of each literal in $S$; we say a literal $q$ is \emph{inconsistent} with $S$ if $q \in \neg \cdot S$; two sets $R$ and $S$ of literals are \emph{inconsistent} if their intersection is nonempty, i.e.\ $R \cap \neg \cdot S \neq \emptyset$; $S$ is \emph{inconsistent} if it is inconsistent with itself, i.e.\ some literal and its complement both appear in $S$; $S$ is \emph{consistent} if it is not inconsistent.

Given a program $\mathrm{P}$, a \emph{partial interpretation} $I$ is a consistent set of literals that are formed from the atomic formulas in the Herbrand base of $\mathrm{P}$. A partial interpretation $I$ is said to be \emph{total} if for any ground atomic formula $p$ in the Herbrand base either of its literal is contained in $I$. A literal $p$ is \emph{true in $I$} if $p \in I$ and is \emph{false in $I$} if its complement is in $I$. This notion can be straightforwardly extended to conjunctions in the normal way.

An instantiated rule of a program $\mathrm{P}$ is \emph{satisfied} in a (partial or total) interpretation $I$ if the head (which is a ground atomic formula) is true in $I$ or some subgoal in the body of that rule is false in $I$; the rule is \emph{falsified} in $I$ if the head is false and all subgoals are true. Moreover, if the head of the rule is false in $I$, but no subgoal is false in it then the rule is said to be \emph{weakly falsified} in $I$. Thus, a rule is weakly falsified if it is falsified, but not vice versa.

A \emph{total model} of a program $\mathrm{P}$ is a total interpretation that satisfies all instantiated rules of $\mathrm{P}$. A \emph{partial model} of $\mathrm{P}$ is a partial interpretation that can be extended to a total model of $\mathrm{P}$.
\ \medskip\\
\noindent\textbf{Remarks.}
\begin{enumerate}[label=(\alph*)]
%
\item The reader may think of a partial interpretation as containing incomplete information: The positive (or negative) literals in it are considered to be true (or false, respectively) atomic facts, and the truth values of the rest of the ground atomic formulas are considered unknown, or unspecified, at least ``at present.''
%
\item Since interpretations are sets, it is natural to consider one to be contained in another if the former is a subset of the latter.
%
\item Note that a partial model $I$ of a program $\mathrm{P}$ may not satisfy some instantiated rules of it, but $I$ can be extended -- adding (possibly no) literals to it -- so that all rules are satisfied. Of course, this cannot be done if $I$ falsifies every instantiated rule; if $I$ only weakly falsifies some rules, then there is still hope for this extension; and \emph{if $I$ does not weakly falsify any instantiated rule, then $I$ is a partial model of $\mathrm{P}$}.
%
\end{enumerate}

\subsection{Unfounded Sets and Well-Founded Partial Models}
Having described the preliminaries in the previous subsection, we explain the notions of unfounded sets and well-founded partial models, which are the main part of [\ref{VanGelder}].

\subsubsection{Unfounded Sets}
Given a program $\mathrm{P}$, let $H$ denote the Herbrand base of $\mathrm{P}$, and assume $I$ is a partial interpretation of $\mathrm{P}$. We say $A \subseteq H$ is an \emph{unfounded set} of $\mathrm{P}$ with respect to $I$ if this condition is true for each ground atomic formula $p \in A$: Either there is no instantiated rule of $\mathrm{P}$ for $p$ or, for each instantiated rule $r$ of $\mathrm{P}$ whose head is $p$, at least one of the following statement is true
\begin{enumerate}[label=(\arabic*)]
%
\item Some (positive or negative) subgoal of the body is false in $I$,
%
\item Some positive subgoal of the body occurs in A.
%
\end{enumerate}
A literal that makes (1) or (2) above true is called a \emph{witness of unusability} for rule $r$ (with respect to $I$).
\medskip\\
\textbf{Remark.} The reader may regard $I$ as a collection of known facts about $\mathrm{P}$ thus far. Rules for which (1) holds cannot be used to derive more facts since its body is false. Rules for which (2) holds have unfoundedness in them: To derive a fact from an unfounded set $A$, i.e.\ to decide the truth value of an atomic formula in $A$, the truth values of some (other) atomic formulas must be decided first.

An example to illustrate the above definition is the following program
\[
\begin{array}{lll}
p(a) & \leftarrow & p(c), not \ p(b) \cr
p(b) & \leftarrow & not \ p(a) \cr
p(e) & \leftarrow & not \ p(d) \cr
p(c) & \ & \ \cr
p(d) & \leftarrow & q(a), not \ q(b) \cr
p(d) & \leftarrow & q(b), not \ q(c) \cr
q(a) & \leftarrow & p(d) \cr
q(b) & \leftarrow & q(a)
\end{array}
\]
The set $\{p(d), q(a), q(b), q(c)\}$ is unfounded with respect to $\emptyset$: There is no instantiated rule for $q(c)$ to decide its truth value; for each of the remaining atomic formulas, condition (2) applies. In contrast, the set $\{p(a), p(b)\}$ is not unfounded because the interdependency between $p(a)$ and $p(b)$ is through negation; as soon as either of them is declared false, the other must be true.

The reader should note that unfounded sets are closed under union. Thus, it is valid to speak of the \emph{greatest unfounded set} of $\mathrm{P}$ with respect to $I$, denoted $\mathbf{U}_\mathrm{P}(I)$, which is defined to be the union of all sets that are unfounded with respect to $I$.
\medskip\\
\textbf{Remark.}
\begin{enumerate}[label=(\alph*)]
%
\item Let $\mathrm{P}$ be a program and $R$ be a set of literals, and let $A$ be an unfounded set of $\mathrm{P}$ with respect to $R$, then for any subset $S \subseteq A$, $A - S$ is unfounded with respect to $R \cup \neg \cdot S$. In other words, if we deduce that certain facts $S$ are in an unfounded set $A$ and add their complements to $R$, the other unfounded atomic formulas remain unfounded.
%
\item Let $I$ be a partial interpretation consisting of a set of positive literals $Q$ and a set of negative literals $\neg \cdot S$. If $I$ does not weakly falsify any instantiated rule of program $P$, then $S$ is an unfounded set with respect to $Q$.
%
\end{enumerate}

\subsubsection{Well-Founded Partial Semantics}
In this section we introduce a (possibly transfinite) sequence of \emph{transformations} on sets of literals in which its limit defines the well-founded semantics. Recall that a transformation $\mathbf{T}$ is \emph{monotonic} if $\mathbf{T}(I) \subseteq \mathbf{T}(I')$ whenever $I \subseteq I'$.

Given a program $\mathrm{P}$ and an interpretation $I$, the three transformations $\mathbf{T}_\mathrm{P}, \mathbf{U}_\mathrm{P}, \mathbf{W}_\mathrm{P}$ are defined as
\begin{enumerate}[label=(\arabic*)]
%
\item $\mathbf{T}_\mathrm{P}(I)$ consists of all positive literals $p$ such that there is an instantiated rule $r$ of $\mathrm{P}$ where $p$ is the head and each (positive or negative) subgoal literal in the body of $r$ is true in $I$.
%
\item $\mathbf{U}_\mathrm{P}(I)$ is the greatest unfounded set of $\mathrm{P}$ with respect to $I$, as introduced in the previous section.
%
\item $\mathbf{W}_\mathrm{P}(I) = \mathbf{T}_\mathrm{P}(I) \cup \neg \cdot \mathbf{U}_\mathrm{P}(I)$.
%
\end{enumerate}
Let a program $\mathrm{P}$ be given and let $H$ denote its Herbrand base. Also, let $\alpha$ range over all countable ordinals. Then the sets $I_\alpha \subseteq H$ are defined recursively by
\begin{enumerate}[label=(\arabic*)]
%
\item $I_0 := \emptyset$.
%
\item For limit ordinal $\alpha > 0$, $I_\alpha := \bigcup_{\beta < \alpha} I_\beta$.
%
\item For successor ordinal $\alpha = \beta + 1$, $I_\alpha := \mathbf{W}_\mathrm{P}(I_\beta)$.
%
\end{enumerate}
Moreover, denote $I^\infty := \bigcup_{\alpha} I_\alpha$. Note that $I^\infty \subseteq H$ also.
\medskip\\
\textbf{Remarks.}
\begin{enumerate}[label=(\alph*)]
%
\item Note that by definition, if a literal $p$ is in $I^\infty$ then the smallest ordinal $\alpha$ for which $p$ appears in $I_\alpha$ is a successor ordinal.
%
\item The sequence of $I_\alpha$'s as defined above is a monotonic sequence of partial interpretations, i.e.\ they are consistent sets of literals.
%
\item By classical results of Tarski, $I^\infty$ is the least fixed point of the transformation $\mathbf{W}_\mathrm{P}$. Since $H$ is countable, $I^\infty = I_\alpha$ for some countable ordinal $\alpha$. The smallest ordinal $\alpha$ for which $I^\infty = I_\alpha$ in the sequence of $I_\alpha$'s is the \emph{closure ordinal} for that sequence. Recall that if $\mathrm{P}$ contains no function symbols then $H$ is finite, and hence the closure ordinal must be finite.
%
\end{enumerate}
We then let $I^\infty$, the fixed point of $\mathbf{W}_\mathrm{P}$, to represent the \emph{well-founded semantics} of program $\mathrm{P}$, so that every positive (or negative) literal in $I^\infty$ denotes that its atomic formula is true (or false, respectively) in $\mathrm{P}$, and that atomic formulas that are missing in $I^\infty$ have no truth values or, ``unknown.''
\medskip\\
\textbf{Remarks.}
\begin{enumerate}[label=(\alph*)]
%
\item \emph{Each $I_\alpha$ as defined above does not weakly falsify any instantiated rule of $\mathrm{P}$ and hence is a partial model of $\mathrm{P}$.}
%
\item We then say that $I^\infty$ is the \emph{well-founded partial model} and, if $I^\infty$ is a total interpretation, i.e.\ if for every $p \in H$ either $p$ or $\neg p$ is in $I^\infty$, then we say $I^\infty$ is a \emph{well-founded model}.
%
\item Thus, every Horn program has a well-founded model $I^\infty$, which is the minimum model in the sense of Van Emden and Kowalski [\ref{VanEmden}], that is, its positive Iiterals are contained in every Herbrand model.
%
\end{enumerate}
\subsection{Comparisons}
Having introduced the well-founded semantics, we discuss below some comparisons between it and other approaches.

Clark introduced the completed program as a way of formalizing the notion that facts not inferable from the rules in the program were to be regarded as false [\ref{Clark}]. The idea is to collect all rules having the same head predicate into a single rule whose body is a disjunction of conjunctions, then replace the symbol $\leftarrow$ (if) by $\leftrightarrow$ (if and only).

The original ``logical consequence'' approach essentially declares that only conclusions that are logical consequences (in the classical, 2-valued sense) of the completed program should be inferred. When the completed program is consistent, this approach implicitly defines a 3-valued interpretation: Assign value true to instantiated atoms that are true in all (2-valued, not necessarily Herbrand) models of the completed program, false to instantiated atoms that are false in all models, and $\bot$ (unknown) to all other instantiated atoms. By [\ref{fitting2}], the completion of every program has a (unique) minimum 3-valued Herbrand model. Fitting suggests that this model be taken for the semantics of the program, which we call the \emph{Fitting model}. 

Having introduced several different approaches, we make the following comparisons between well-founded semantics and the other semantics:
\medskip\\
\textbf{Remark.}
\begin{enumerate}[label=(\alph*)]
%
\item The Fitting model is a subset of $I^\infty$.
%
\item If program $\mathrm{P}$ has a well-founded total model, then that model is the unique stable model.
%
\item The well-founded partial model of $\mathrm{P}$ is a subset of every stable model of $\mathrm{P}$.
%
\item If program $\mathrm{P}$ is locally stratified then it has a well-founded model, which is identical to the \emph{perfect model}.\footnote{See [\ref{prz}] for definition of perfect models.}
%
\end{enumerate}