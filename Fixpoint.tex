The least fixpoint query we discuss here refer to domain relational calculus with least fixpoint operator extension, as described in textbook\cite{Abiteboul1}, which is intuitively least fixpoint logic under the context of relational database theory. We first briefly describe its syntax and semantics in the first section, where we skip the part belong to relation calculus and focus on the definition of least fixpoint operator. Then we will put our attention on the discuss of its expressive power and data complexity in second one. \\
We are also going to cover several important theorems related to these topics and probably contain some simple sketch of proof.

\subsection{Syntax and semantics}
The syntax and semantics of least fixpoint query are mostly the same as domain relational calculus except the added least fixpoint operator. To define the notion of fixpoint operator, we need first introduce the notion of operator. \\
An operator $\Phi: \textit{P}(A^k)\rightarrow \textit{P}(A^k)$ is a mapping defined on the set of all k-ary relations on the universe $A$ of structure $\textbf{A}$, where universe and structure are basic concepts of finite model theory (for detail, check section 2.2 in \cite{kolaitis1}). Under context of domain relational calculus, we could consider the structure as database instance and its universe as the active domain of that database instance. Then if a k-ary relation $P$ satisfy $P=\Phi(P)$, then $P$ is said to be a fixpoint of $\Phi$. And if a fixpoint $P^*$ of $\Phi$ satisfy $P^*\subseteq P$ for all fixpoint $P$ of $\Phi$, then $P^*$ is said to be a least fixpoint of $\Phi$.\\
Intuitively, an operator does not necessary have a fixpoint point. So we need to add some constraints to the operator we are discussing to ensure the least fixpoint's existence. And these contains are addressed by a theorem called Knaster-Tarski Theorem\cite{Tarski}. \\

Before describing what the Knaster-Tarski Theorem says, we need to define some basic concepts first. For every ordinal $\alpha$, we define 
$$\Phi^{\alpha}=\Phi(\cup_{\beta<\alpha} \Phi^{\beta})$$
where $\Phi^0=\emptyset$. And we say an operator $\Phi$ is monotone if 
$$\Phi\models(\forall P_1,P_2\in P(A^k)(\Phi(P_1)\subseteq\Phi(P_2)))$$
Then the Knaster-Tarski Theorem basically states that if operator $\Phi$ is monotone, then $\Phi$ has a least fixed point $\textbf{lfp}(\Phi)$ and there exists an ordinal $s\leq |A|^k$ (assume active domain A is finite) such that $\textbf{lfp}(\Phi)=\Phi^{\infty}=\Phi^s=\Phi^{\delta}$, for all $\delta > s$.\\
After defined the notion of least fixpoint of operator, we are now going to discuss how to argument relational calculus with it. A natural approach is to express the required monotone operator as relational calculus expression and then use its least fixpoint as an extension. But M. Ajtai and Y. Gurevich (1987,\cite{Ajtai}) had shown the problem of monotonicity of a first-order formula is undecidable. So instead they introduced a new notion called positivity, which can implies monotonicity but is decidable. The notion of positivity can be stated as follows: a first order logic formula $\varphi(S)$ that contains a k-ary relation $S$ is said to be positive in $S$ if every occurance of $S$ in $\varphi(S)$ is under an even number of negation. And if a first order logic formula $\varphi(S)$ is positive in $S$, then it (under context of relational database theory) is monotone on all database instances under the database schema where the formula is defined.\\ 
Now after restrict the relational calculus formula we used to define operator to be positive, each formula is now associated with a least fixed point. And according to the Knaster-Tarski Theorem, we also it can be computed by repeatedly apply the corresponding operater. We can now define the syntax and semantics of our least fixpoint query.(for formal definition of least point logic, see section 2.6 in \cite{kolaitis1})\\

\begin{description}

\item[Syntax:]
Let $\varphi(\textbf{x},\textbf{y}, S_1,S_2,..,S_n, T)$ be a relational calculus formula, where $\textbf{x}=(x_1, x_2,...,x_k)$, $\textbf{y}=(y_1,y_2,...,y_m)$ are variables, $S_1,S_2,..,S_n, T$ are relation symbols and $\varphi$ is positive in $T$. Let $\textbf{u}=(u_1, u_2,...,u_k)$ be a k-tuple of variable not occuring in $\varphi$. Then $[\textbf{lfp}\,T\textbf{x}.\varphi](u)$ is a least fixpoint formula. And more specifically, if $\varphi$ only consists of $\textbf{x}$ and $T$, then we can write the corresponding least fixpoint formula compactly as $\mu_T(\varphi)(u)$. We call $\mu_T$ the least fixpoint operator. Farther more, the queries that are definable by least fixpoint formula are called least fixpoint query.\\

\item[Semantics:]
Given database instance $A$ and a tuple $a$, then $A, a \models [\textbf{lfp}\,T\textbf{x}.\varphi](u)$ if $A, a \models R(u)$ where $R$ is interpreted as a relation such that $R=\textbf{lfp}(\Phi)$, where $\Phi$ is a $k$-ary operator such that
$$ \Phi(T) = \{\textbf{x}\in adom(A)^k: A\models \varphi(\textbf{x},\textbf{y},S_1,S_2,..,S_n,T)\}$$ 

\end{description}

In the following section, we only consider the least fixpoint formulas that contain $\mu_T(\varphi)(u)$. In other words, formulas that do not allow parameters to appear inside the input formula of least fixpoint operator. We do it for simplicity to avoid the complexity of nested least fixpoint queries.

\subsection{Expressive Power and Data Complexity}

In this section, we are going to discuss the expressive power of least fixpoint query. As an extension to relational calculus query, least fixpoint query has higher expressive power. We can illustrate it via a P-complete decision problem called \textbf{Path Systems}\cite{Cook}. Path Systems is a decision problem not definable in relational calculus but definable in our least fixpoint query. \\

\begin{description}
\item[Path Systems:]
Let $F$ be a unary relation representing a set of formulas, $A$ be a unary relation representing a set of axioms, and $R$ be a ternary relation representing a set of rules of inference. Then for a database instance $S = (F, A, R)$ , we say a formula $f\in F$ is theorem of $S$ if $f\in A$ or there exist other two theorems $g$ and $h$ of $S$ and a tuple $(f, g, h)$ in $R$. Then the Path System problem can be stated as follows: Given such an database instance $S$, and a formula $f\in F$, is $f$ a theorem of $S$?
\end{description}

\begin{claim}
Path Systems Problem is definable in least fixpoint query.
\end{claim}

It's very straightforward to show above decision problem is definable in our least fixpoint query, by directly translate its definition into following boolean least fixpoint formula:
$$ \mu_T(A(x)\lor (\exists g \exists h)(T(g)\land T(h) \land R(x, g, h)))(f) $$
And since we reduced a P-complete problem to a specific least fixpoint formula, we also proved that the data complexity of our least fixpoint query is at least P, P-hard in other word.\\

Proving this Path System problem is not definable in relational calculus is relatively difficult. To do that, we need to introduce a useful notion, called Ehrenfeucht-Fra\"iss\'e games, which is commonly used for proving certain problem is undefinable in certain logic. The method of Ehrenfeucht-Fra\"iss\'e games has been proven to be an effective tool in studying the limitation of expressive power of first-order logic and many other logic of higher expressive power.

\begin{description}
\item[Ehrenfeucht-Fra\"iss\'e games]
Given two database instances $A$ and $B$ of the same database schema. A $r$-move Ehrenfeucht-Fra\"iss\'e game on $A$ and $B$ has two players, Spoiler and Duplicator. They play the game in turn. In each turn $1\leq i \leq r$, the Spoiler first picks an element $a_i$ from $adom(A)$ (or $b_i$ from $adom(B)$), then the Duplicator picks an element $b_i$ from $adom(B)$ (or $a_i$ from $adom(A)$ if Spoiler picked from $adom(B)$). After $r$ turns, a picking sequence is generated $(a_1, b_1),...,(a_i,b_i),...,\\(a_r, b_r)$, which is called a run of $r$-move Ehrenfeucht-Fra\"iss\'e game. Then the Duplicator wins this run if the mapping $f$ of form
$$ f: a_i\rightarrow b_i, 1\leq i \leq r$$
is a partial isomorphism between $A$ and $B$, otherwise the Spoiler wins this run. And if the Duplicator has a winning strategy to allow it to win every run, the Duplicator wins the $r$-move Ehrenfeucht-Fra\"iss\'e game on $A$ and $B$, otherwise the Spoiler wins.
\end{description}

The relation between the definability of a decision problem in relational calculus and Ehrenfeucht-Fra\"iss\'e games is addressed by a famous method, which is called \textbf{The Method of Ehrenfeucht-Fra\"iss\'e games} (for formal definition and proof, check \cite{kolaitis1}). We just (informally) describe it under the context of relational database theory.

\begin{description}
\item[The Method of Ehrenfeucht-Fra\"iss\'e games for Relational Calculus]
\end{description}
Let $Q$ be a boolean query defined on database schema $S$, then $Q$ cannot be expressed as a relational calculus expression on $S$ if for every positive integer $r$ there exist two database instances $A_r, B_r$ on $S$ such that
\begin{enumerate}
    \item $Q(A_r)=1\land Q(B_r)=0$.
    \item The winner of $r$-move Ehrenfeucht-Fra\"iss\'e game on $A_r$ and $B_r$ is Duplicator.
\end{enumerate}
Now let us prove the Path System problem cannot be expressed by a relational calculus expression.

\begin{claim}
Path Systems problem is not definable in relational calculus.
\end{claim}

\begin{proof}
Let two database instances $I$ and $J$ of the same schema be the following:
$$I=(\{F(f), F(k)\}, \{A(g),A(h)\},\{R(f,g,h)\})$$
$$J=(\{F(f), F(k)\}, \{A(g),A(h)\}, \{R(k,g,h)\})$$
Then for a query $Q$ asking whether $f$ is a theorem we have $Q(A)=1 \, \text{and} \, Q(B)=0$. Then the winning strategy of the Duplicator is: for every $f,k,g,h$ that Spoiler picks, Duplicator picks $k,f,g,h$ correspondingly. Notice that this strategy makes Duplicator win $r$-move Ehrenfeucht-Fra\"iss\'e games for every positive integer $r$. So we can conclude that query $Q$ is not definable in relational calculus.
\end{proof}

After prove Path System problem is not definable in relational calculus but definable in our least fixpoint query, we therefore proved the least fixpoint operator we introduced is a effective extension to relational calculus.\\

Then, after talk about the improved expressive power of least fixpoint query. Now let's discuss the data complexity. As we have seen before, by reduce Path System problem to least fixpoint query, we saw least fixpoint query's data complexity is P-hard. Now we are going to see it is actually P-complete.\\

\begin{claim}
Data Complexity of least fixpoint query is P-complete.
\end{claim}
\begin{proof}
Recall in the syntax of least fixpoint query, each least fixpoint formula is required to be positive. This property lead to the monotonicity of its corresponding k-ary operator $\Phi$. And also recall the Knaster-Tarski Theorem which states given a database I, we have $\textbf{lfp}(\Phi)=\Phi^s$ where $s<adom(I)^k$. So the number of iterations takes for the operator $\Phi$ to reach least fixed point is in polynomial of size of database. Then since the operator $\Phi$ is defined by ordinary relational calculus formula, and the data complexity of relational calculus is P. So each iteration also takes polynomial. Combine them together, we have data complexity of the original least fixpoint formula in P. Finally, as we have seem Path System problem which is P-complete can be reduced to a least fixpoint query , the data complexity of least fixpoint query is also P-complete. 
\end{proof}

After describe the expressive power and data complexity of least fixpoint query, one thing also worth mentioned is: Although least fixpoint query is more powerful than relational calculus and it contains P-complete query like Path System Problem, it still have serious limitation. There are still many simple problems undefinable in least fixpoint query. E.g. the Even Cardinality problem\cite{kolaitis1}. From which we see even with its P-complete data complexity there are still many problem in Playing outside of least fixpoint query's expressive power.