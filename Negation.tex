We have discussed the extension of the calculus with recursive queries 
using the least fixpoint semantics. Another desirable feature that is absent from 
basic logic programming is negation. There are two forms of negation: strong (``classical'') 
negation to express falsity and negation as failure to express non-truth. The 
latter is a weaker version of negation: instead of concluding 
$\neg p$ if $p$ is true, we conclude $not \: p$ if $p$ is not known to be true. 
Classical negation can be problematic in logic programming that computes 
recursively enumerable relations because their complements cannot be computed [1]. 
Going forward, we will instead focus on logic programs with negation as failure. 

The theory behind negation as failure arises from nonmonotonic reasoning, which states 
the possibility of retracting a conclusion given new information [5]. For instance, 
consider the following logic program with negation as failure:
\begin{align}
    \begin{split}
        & p(a). \hspace{1em} p(b). \hspace{1em} q(a). \\
        & r(X) \leftarrow p(X), not \: q(X).
    \end{split}
\end{align}
we can conclude $r(b)$ from 
program (1), but this conclusion is invalid if fact $q(b)$ is added. Logic programming with 
negation can be translated into two nonmonotonic formalisms: default and autoepistemic logic. 
Autoepistemic logic couples each set $A$ of axioms with the operator $L$ to express the idea that 
$A$ is believed to be true, or $LA$. Conversely, the expression $not \: A$ is analogous to the 
autoepistemic formula $\neg LA$. We can rewrite program (1) using autoepistemic logic:
\begin{align}
    \begin{split}
        & p(a). \hspace{1em} p(b). \hspace{1em} q(a). \\
        & r(a) \leftarrow p(a), \neg Lq(a). \\ 
        & r(b) \leftarrow p(b), \neg Lq(b).
    \end{split}
\end{align}
where each ground negative literal $\neg B$ becomes $\neg LB$. It follows from the 
autoepistemic theory that program (2) has a unique stable expansion, whose atoms 
form the intended model. Observe that the program is 
grounded to remove its variables before the translation. Autoepistemic logic provides an intuitive 
representation of negation as failure as the absence of belief. 

A logic program with negation can also be translated into default logic, where the head 
of each rule becomes the conclusion, the conjunction of positive literals becomes the premise, 
and each negative literal becomes a justification $M \neg A$ (which reads as ``it is consistent to assume $\neg A$''). 
Accordingly, program (1) corresponds to the following default logic:
\begin{align}
    \begin{split}
        \frac{}{p(a)} \hspace{1cm} \frac{}{p(b)} \hspace{1cm} &\frac{}{q(a)} \\ 
        \frac{p(X) \: : \: M \neg q(X)}{r(X)}&
    \end{split}
\end{align}
Even though default logic allows the use of variables in a default, each variable is simply 
the syntactic sugar for a set of its ground instances. Thus, programs (2) and (3) are essentially 
the same: asserting $q(a)$ and $q(b)$ are not believed is synonymous to asserting that it is consistent to assume 
$\neg q(X)$. Scoping a logic program within the perimeter of its ground instances is a key step 
towards the first definition of a stable model. 

Adding negation to Horn logic programs, however, may result in the existence of multiple minimal 
Herbrand models. 
Program (1) has two minimal models, $\{p(a),p(b),\\q(a),r(b)\}$ and $\{p(a),p(b),q(a),q(b)\}$. 
Observe that the latter model is considered ``bad'' because $q(b)$ is not supposed to be true. 
The major challenge when extending logic programming with negation is to invent a semantics that 
captures the correct minimal model of a program. 
Two approaches are proposed to preserve the notion of a single intended model for a logic program 
in the presence of negation. 
Apt, Blair, and Walker [6] introduce a stratified class of logic programs in which recursion and negation are 
mutually exclusive. In this case, there exists a single \emph{perfect model} for every stratified program. 
This technique, however, comes at the expense of restricting the semantics of logic programs with negation. 
The second approach removes this restriction by assigning to an arbitrary program a single 3-valued 
model using the well-found semantics. We explain this semantics in more detail in Section 5. 
On the other hand, we also have what is called the stable model semantics, which 
assigns a collection of intended models rather than a single one to a logic program with negation. 
Even though this property violates the notion of a single model that we are trying to salvage when 
extending logic programming with negation, it makes the stable model semantics a useful 
computational technique for solving problems with many solutions. 
The following section discusses the semantics and applications of stable models. 