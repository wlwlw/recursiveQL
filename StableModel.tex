There is a large body 
of study on the concept of a stable model using different interpretations. To put our paper into 
context, we first cover the classic definition of a stable model introduced in Gelfond and Lifschitz 
\cite{gelfond}, 
which provides an intuitive mechanism using autoepistemic logic to model logic programs with negation. 
Then, we describe the modern  
technique of circumscription to obtain a stable model by translating a logic program 
into a formula of second-order logic \cite{lee}. The second interpretation of stable models gives semantics 
to nontraditional constructs found in answer set programming. Section 4.5 examines the role of 
the stable model semantics in logic programming and compares it with the traditional Horn logic. Finally, 
we illustrate the application of stable models in ASP. 

\subsection{Gelfond-Lifschitz Reduct}
The most fundamental definition of a stable model is based on the Gelfond-Lifschitz transformation 
process \cite{gelfond} of removing negation from a logic program $\Pi$. 
We first ground $\Pi$ so that each variable occurrence in $\Pi$ is replaced by its ground instances. 
The reduct of $\Pi$ relative to a set $M$ of its ground atoms is obtained by the following two 
elimination procedures:
\begin{enumerate}[label=(\roman*)]
    \item Eliminating each rule with a negative literal $not \: A_i$ with $A_i \in M$, and 
    \item Eliminating all the negative literals from the bodies of the remaining rules.
\end{enumerate}
The reduct of $\Pi$ follows the intuition behind a stable expansion in 
autoepistemic logic. Here, $M$ is considered the possible set of beliefs that a rational 
agent might hold, given $\Pi$ as his premises \cite{gelfond}. If a rule $R$ contains a literal $not \: A_i$ 
such that $A_i \in M$, then we cannot establish $not \: A_i$. Hence, $R$ is false in $M$ and can be 
removed from $\Pi$, as stated in procedure (i). Otherwise, if $A_i \not \in M$, we can conclude 
$not \: A_i$, and this literal is trivial. Removing $not \: A_i$ from the body of $R$ (as in procedure (ii)) 
does not change the semantics of the program. Since the reduct of $\Pi$ no longer 
contains any negation, it has a unique minimal Herbrand model.  
It follows that $M$ is a stable model of $\Pi$ if it coincides with such minimal Herbrand model.

Let $\Pi$ be program (1), where each rule is replaced by its ground instances 
\begin{align*}
    & p(a). \hspace{1em} p(b). \hspace{1em} q(a). \\
    & r(a) \leftarrow p(a), \neg q(a). \\ 
    & r(b) \leftarrow p(b), \neg q(b).
\end{align*}
Let $M = \{p(a),r(a)\}$. Then the reduct of $\Pi$, $\Pi _M$, is 
\begin{align*}
    & p(a). \hspace{1em} p(b). \hspace{1em} q(a). \\
    & r(a) \leftarrow p(a). \\ 
    & r(b) \leftarrow p(b).
\end{align*}
The minimal Herbrand model of $\Pi _M$ is $\{p(a),p(b),q(a),r(b),r(a)\}$, which does not 
match $M$. Hence, $M$ is not a stable model. Let us pick a different 
$M = \{p(a),p(b),q(a),\\r(b)\}$. Then $\Pi _M$ is
\begin{align*}
    & p(a). \hspace{1em} p(b). \hspace{1em} q(a). \\
    & r(b) \leftarrow p(b). 
\end{align*}
In this case, the least model of $\Pi _M$ coincides with $M$, and $M$ is stable.

The Gelfond-Lifschitz reduct provides an intuitive way to represent a logic 
program with negation by removing the negation itself. However, this definition 
of a stable model has its limitations. First of all, the reduction process relies 
on the grounding of a logic program, which can prove impractical in the presence of 
function symbols. In this case, there are infinitely many ground instances. 
Secondly, the reduct of a logic program serves only to verify whether a predetermined 
set $M$ of ground atoms is stable or not. It is more convenient to have a systematic 
technique to generate a stable model without guessing. The next definition of a stable 
model removes the hassle of grounding through the use of circumscription.

\subsection{Circumscription}
Similar to autoepistemic and default logic, circumscription is another variant of 
nonmonotonic reasoning that exerts minimization on the extension of predicates in 
second-order logic \cite{lifschitz2}. We have been using minimal Herbrand models to describe the semantics 
of logic programs. Yet, a logic program can also be seen as a representation of a set of 
first-order formulas called \emph{program completion}. For instance, program (1) 
represents the following formula
\begin{align}
    \forall x (p(x) \leftrightarrow x = a \lor x = b) \land 
    \forall x (q(x) \leftrightarrow x = a) \land 
    \forall x (p(x) \land \neg q(x) \rightarrow r(x))
\end{align}
Given a program completion $F$, we first replace all the occurrences of negative literals 
with a new symbol $p'$. The circumscription of $F$, $CIRC[F]$, is denoted as 
\begin{align}
    F \land \neg \exists u((u < p) \land F(u)) \land \forall x (p'(x) \leftrightarrow p(x))
\end{align}
where $p$ is the set of all predicate constants in $F$, $u$ is the set of distinct 
predicate variables of the same length, and $F(u)$ substitutes the variables $u$ for 
the constants $p$. The second conjunctive term of $CIRC[F]$ expresses the minimal 
condition of the predicates $p$ in the sense that we cannot further reduce $p$ without 
invalidating $F$. The third conjunctive term asserts that each new predicate is 
equivalent to its old predicate.

For example, let $F$ be formula (4) where each negative literal is renamed.
\begin{align*}
    \forall x (p(x) \leftrightarrow x = a \lor x = b) \land 
    \forall x (q(x) \leftrightarrow x = a) \land 
    \forall x (p(x) \land \neg q'(x) \rightarrow r(x))
\end{align*}
then $CIRC[F]$ is 
\begin{multline*}
    \forall x (p(x) \leftrightarrow x = a \lor x = b) \land 
    \forall x (q(x) \leftrightarrow x = a) \land 
    \forall x (p(x) \land \neg q'(x) \rightarrow r(x)) \\ 
    \land \neg \exists uvz (
        \forall x (u(x) \leftrightarrow x = a \lor x = b) \land 
        \forall x (v(x) \leftrightarrow x = a) \land 
        \\ \forall x (u(x) \land \neg q'(x) \rightarrow z(x)) 
    ) 
    \land \forall x (q'(x) \leftrightarrow q(x))
\end{multline*}

\subsection{Modified Circumscription}
One advantage that circumscription has over the Gelfond-Lifschitz reduct 
is the omission of grounding as the preliminary step to determine a stable 
model. Ferraris et al. \cite{lee} propose a modified circumscription of 
a program completion $F$, $SM[F]$, which is denoted as 
\begin{align}
    F \land \neg \exists u((u < p) \land F^*(u)) 
\end{align}
where $u$, $p$ are the same symbols used for the original circumscription formula (5), 
and $F^*(u)$ 
replaces each predicate constant $p_i$ that is not a negation with 
the corresponding predicate variable $u_i$. Hence, formula (4) could 
have been simply rewritten as 
\begin{multline*}
    \forall x (p(x) \leftrightarrow x = a \lor x = b) \land 
    \forall x (q(x) \leftrightarrow x = a) \land 
    \forall x (p(x) \land \neg q'(x) \rightarrow r(x)) \\ 
    \land \neg \exists uvz (
        \forall x (u(x) \leftrightarrow x = a \lor x = b) \land 
        \forall x (v(x) \leftrightarrow x = a) \land 
        \\ \forall x (u(x) \land \neg q(x) \rightarrow z(x)) 
    ) 
\end{multline*}

It follows 
that any Herbrand interpretation that satisfies $SM[F]$ is a stable model. 
Note that $SM[F]$ is identical to $CIRC[F]$ if $F$ is negation-free. 
$SM[F]$ differs from $CIRC[F]$ by not replacing  
negated predicates with auxiliary predicates and adding an additional 
conjunctive term to relate the two. In addition to the omission of 
grounding, the modified version of circumscription 
extends the stable model semantics to programs with nontraditional constructs 
(such as choice rules, disjunctive rules, and constraints) used in ASP as 
long as they are expressible in first-order logic. The Gelfond-Lifschitz reduct, 
on the other hand, is only applicable to programs in the form of Horn clauses.

\subsection{Properties}
Because of its intuitive definition, the stable model semantics is commonly 
accepted as the standard for logic programming with negation. The 
following are the properties of the stable model semantics:
\begin{enumerate}[label=(\roman*)]
    \item Any stable model of a logic program $\Pi$ is its minimal Herbrand model \cite{gelfond}. 
    \item A well-founded model of $\Pi$ defines its unique stable model \cite{gelfond}. 
    \item If $\Pi$ is stratified, then there is a unique stable model that coincides 
    with the perfect model of $\Pi$ \cite{gelfond}.
    \item Any stable model of $\Pi$ is a subset of the set of head atoms in $\Pi$ \cite{ferraris}.
    \item $\Pi$ might have several stable models, or no stable models at all. For 
    instance, the program consisting of one rule $p \leftarrow not \: p$ has no 
    stable models, while the program with two rules $p \leftarrow not \: q$ 
    and $q \leftarrow not \: p$ has two stable models, $\{p\}$ nad $\{q\}$. The 
    well-founded semantics helps solve this shortcoming of the stable model 
    semantics by ensuring the existence of a unique stable model for $\Pi$ \cite{fitting}.
    \item If there are several stable models for $\Pi$, they are both minimal 
    and incomparable \cite{fitting}. 
    \item The problem of determining the existence of a stable model for a finite 
    propositional logic program is NP-complete \cite{marek}. This follows from the expressive 
    power of the stable model semantics to represent all decision and search 
    problems in NP. 
\end{enumerate}

\subsection{Stable Logic Programming} 
We have seen that a logic program might have several incomparable stable models, 
which brings into question the use of stable models as the semantics for 
logic programming with negation. Rather 
than trying to reconcile the stable model semantics with Horn logic programming, 
Marek and Truszczynski \cite{marek} argue that the lack of a single intended model 
makes the stable model semantics suitable for representing 
some constraint satisfaction problems whose solutions 
are a finite family of finite sets \cite{marek}. 
They refer logic programming based on the stable model semantics  
as stable logic programming (SLP). A decision problem that is associated with 
stable models is whether or not a finite logic program has a stable model. If we 
encode an instance $I$ of a decision problem $\Pi$ in the class NP as a propositional program 
$P ^I _ \Pi$, then $\Pi$ has a solution for $I$ if and only if $P ^I _ \Pi$ 
has a stable model \cite{marek}. Hence, the problem of the existence of a stable model is 
NP-complete, as stated in property (vii). In comparison to Horn logic programming 
which can specify any recursively enumerable set, SLP is only able 
to solve NP problems \cite{marek}, which, however, are among the most important computational 
problems in computer science.

Unlike Horn logic programming, SLP disallows the use of function symbols in favor of 
a finite set of ground instances and stable models. The syntax of SLP emulates 
that of DATALOG with negation. The absence of function symbols, however, restricts the 
use of recursion in SLP. In fact, recursion is defined in terms of intensional predicates, 
and not in terms of function symbols additionally used in Horn logic programming. Instead 
of constructing stable models recursively, SLP first specifies a set of potential candidate 
stable models, and then eliminates those that do not satisfy certain constraints. This 
generate-and-test methodology is the basis of answer set programming. 

One key feature of logic programming is the separation of logic from control. The logic 
of a program itself determines the solutions without any execution specifications from 
the programmer. In Horn logic programming, the control of a program is accomplished 
using the \textit{SLD-resolution} to compute the solutions \cite{marek}. Accordingly, SLP also 
has some uniform control mechanisms to process SLP programs and compute their stable models. 
Since there are no function symbols in SLP, it is possible to exhaust all subsets of the 
finite Herbrand base of a SLP program for stable model candidates. Both the 
expressive power to encode NP search problems and the availability of 
a uniform control mechanism make SLP a useful computational tool. One such example is answer set 
programming.

\subsection{Answer Set Programming}
Answer set programming is a form of stable logic programming that solves difficult 
search problems \cite{ferraris}. ASP uses an answer set solver such as \textit{smodels} as a uniform control 
to compute stable models as solutions to a search problem. The frontend of \textit{smodels}, 
called \textit{lparse}, defines a set of constructs to formulate a search problem before 
being analyzed by the answer set solver. 

\subsubsection{Syntax}
Programs 
with multiple stable models usually consist of choice formulas. For example, the rule 
\begin{align}
    \{q(X)\} \: \mathop{:\!\!-} \: p(X).
\end{align}
describes all possible subsets of a given set $p$. 

\textit{lparse} also provides cardinality expressions $l \: \{...\} \: u$ to impose numerical 
bounds on a choice rule, where $l$ is a lower bound and $u$ is an upper bound. We can change 
rule (7) to contain a lower bound for the head atom, 
\begin{align*}
    1 \: \{q(X)\} \: \mathop{:\!\!-} \: p(X).
\end{align*}
to consider only subsets of $p$ with at least one element. 

Another useful ASP construct is the use of disjunction in the head of a rule. The 
rule 
\begin{align*}
    q(X) \: ; \: r(X) \: \mathop{:\!\!-} \: p(X).
\end{align*}
enumerates all possible partitions of a set $p$ into two disjoint subsets $q$ and $r$. 

A conditional literal $\{l:l_1:...:l_n\}$, where $l_i$ are literals, describes a set of 
literals $l$ if $l_1,...,l_n$ are true. For instance, instead of listing all possible 
ground instances of $r(X)$ as $\{r(a),r(b),r(c)\}$, we can take advantage of a conditional 
literal as follows:
\begin{align*}
    &p(a). \hspace{1em} p(b). \hspace{1em} p(c). \\
    &\{r(X) \: : \: p(X)\}.
\end{align*}

So far, we have only discussed programs with negation as failure. ASP also supports 
strong negation in the form of $-p$ to express ``$p$ is false'' as opposed to 
the negation as failure $not \: p$ to express ``$p$ is not known to be true''. Combining 
both kinds of negation $-p \: \mathop{:\!\!-} \: not \: p$ gives the closed world assumption 
stating that ``$p$ is false if there is no evidence that it is true'' \cite{ferraris}. Strong negation 
is useful to represent negative facts in an incomplete state of knowledge.

Finally, a constraint is a rule with an empty head 
\begin{align*}
    \mathop{:\!\!-} \: p.
\end{align*}
which prohibits 
any generation of a literal $p$. Constraints appear in almost every ASP program to weed out 
``bad'' stable models. All of the constructs in ASP are expressible in first-order 
logic \cite{lifschitz1}. Hence, the stable models of an ASP program can be determined using the 
technique of modified circumscription, as described in Section 4.3.

\subsubsection{Generate-and-Test}
The logic of an ASP program follows the generate-and-test methodology introduced in 
stable logic programming. The two steps are first (1) to generate all possible stable models 
for a program, and then (2) to select only those that satisfy additional constraints. Let us 
consider an example ASP program to find Hamiltonian cycles in a directed graph \cite{lifschitz0}. First of all, 
we define a directed graph $G$ with a set of vertices $V$ and a set of edges $E$: 
$vertex(a)$ for all vertices $a \in V$, and $edge(a,b)$ for all directed edges $(a,b) \in E$. Next, 
we generate all potential stable models for this problem using the choice rule 
\begin{align*}
    & \{ \: in(X,Y) \: \} \: \mathop{:\!\!-} \: edge(X,Y). 
\end{align*}
These stable models represent all subsets of edges that could potentially form a Hamiltonian 
cycle. Adding the following constraints 
\begin{align*}
    & \mathop{:\!\!-} \: 2 \: \{in(X,Y) \: : \: edge(X,Y)\}, \: vertex(X). \\
    & \mathop{:\!\!-} \: 2 \: \{in(X,Y) \: : \: edge(X,Y)\}, \: vertex(Y). 
\end{align*}
eliminates subsets of edges that contain pairs of edges starting or ending at the same 
vertex. We further refine the stable models to the problem with these additional clauses: 
\begin{align*}
    & r(X) \: \mathop{:\!\!-} \: in(0,Y), \: vertex(X).\\
    & r(Y) \: \mathop{:\!\!-} \: r(X), \: in(X,Y), \: edge(X,Y). \\ 
    & \mathop{:\!\!-} \: not \: r(X), \: vertex(X).
\end{align*}
Accordingly, each of the final stable models encodes a Hamiltonian cycle wherein every vertex in the graph 
is reachable from an initial 
vertex by a non-empty sequence of $in$ edges.

We have presented two definitions of a stable model using the classic 
Gelfond-\\Lifschitz reduct method 
and the modified circumscription. Even though the stable model semantics has its limitation in 
defining a single intended least model for a logic program, it is suitable for solving 
the type of search problems that involves multiple solutions. One such application is answer 
set programming, which computes stable models as solutions to a search problem. The next 
section discusses an alternative semantics for logic programming with negation using 
a well-founded model.