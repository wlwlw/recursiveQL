\documentclass[11pt]{article}
\usepackage{fullpage,amsthm,amsfonts,amssymb,epsfig,amsmath,times,amsthm}
\usepackage{csquotes}
\usepackage[margin=4cm]{geometry}
\usepackage{enumitem}
\usepackage{titlesec}

\def\ojoin{\setbox0=\hbox{$\bowtie$}%
  \rule[-.02ex]{.25em}{.4pt}\llap{\rule[\ht0]{.25em}{.4pt}}}
\def\leftouterjoin{\mathbin{\ojoin\mkern-5.8mu\bowtie}}
\def\rightouterjoin{\mathbin{\bowtie\mkern-5.8mu\ojoin}}
\def\fullouterjoin{\mathbin{\ojoin\mkern-5.8mu\bowtie\mkern-5.8mu\ojoin}}

% Relational algebra symbols from ftp://reports.stanford.edu/www/dbgroup_only/latex-macros.html
\newcommand{\select}{\mbox{\Large$\sigma$}}
\newcommand{\cross}{\mbox{$\times$}}
\newcommand{\intersection}{\mbox{$\cap$}}
\newcommand{\intersect}{\mbox{$\cap$}}
\newcommand{\union}{\mbox{$\cup$}}
\newcommand{\join}{\mbox{$\Join$}}
\newcommand{\leftsemijoin}{\mbox{$\mathrel{\raise1pt\hbox{\vrule height5pt
depth0pt width0.6pt\hskip-1.5pt$>$\hskip -2.5pt$<$}}$}}
\newcommand{\rightsemijoin}{\mbox{$\mathrel{\raise1pt\hbox{\hskip-1.5pt$>$\hskip -2.5pt$<$\hskip -1.1pt\vrule height5pt
depth0pt width0.6pt}}$}}
\newcommand{\project}{\mbox{\Large$\pi$}}
\newcommand{\Project}{\mbox{$\Pi$}}
\newcommand{\aggregatefn}{\mbox{\Large$G$}}

\newtheorem{theorem}{Theorem} 
\newtheorem{claim}[theorem]{Claim}

\begin{document}

\title{Query Languages with Recursion and Negation}
\author{Liang Wang, Wei-Lin Wu, Tien Ho}
\maketitle

\begin{abstract}
  In this paper, we explore the extension of the query languages we have 
  learned so far with recursion and negation. It is known that 
  recursive queries are not expressible in relational calculus. The least 
  fixpoint semantics helps overcome this limitation by using a fixpoint 
  operator to evaluate recursion. Extending logic programming with negation 
  also adds the complication of a logic program having multiple minimal 
  Herbrand models. We discuss the stable model and well-founded semantics 
  as the two common mechanisms to generalize the least model semantics 
  for both positive and negative programs. In addition, the well-founded semantics 
  solves the issue of some programs having more than one stable model.
\end{abstract}

\section{Introduction} The query languages we have considered so far, including relational calculus, 
relational algebra, and Datalog, all have their expressive limitations. While 
it is proved that neither the calculus nor the algebra is able to express 
recursive queries, Datalog overcomes this limitation yet still lacks support 
for negation. Consider the following query,
\begin{displayquote}
    \textit{``Find all the pairs of vertices in a 
    graph that are not connected.''}
\end{displayquote}
This query is not possible without the use of both recursion and negation. 
In this paper, we explore the extension of relational calculus with 
recursion using the least fixpoint semantics, as well as examine the interpretation  
of logic programs with negation as failure using the stable model 
and well-founded semantics.

Least Fixpoint Query is an extension to relational calculus that embrace recursion. It extends the expressive power of relational calculus by including the computation of least fixed point into its semantics. The least fixed point here is defined as follow: \\
Let $\Phi$ be a mapping between k-ary relations that is expressible in relational calculus. Then if a k-ary relation $P$ satisfy $P= \Phi(P)$, $P$ is said to be a fixed point of $\Phi$. If $P$ also satisfy $P \subseteq P^*$ for all fixed point $P^*$ of $\Phi$, then $P$ is said to be a least fixed point of $\Phi$.
We will talk least fixed point in more detail in later section. Where you can see least fixed point is guaranteed to exists following the syntax of Least Fixpoint Query. And the computation of it is in PTIME, which farther more results the data complexity of Least Fixpoint Query also in PTIME (P-complete precisely). 
In this paper we will also talk about the improved expressive power of Least Fixpoint Query. Where we demonstrate it via an example decision problem called Path Systems, which is expressible in Least Fixpoint Query but not in relational calculus. We will also cover the important theorems behind the reasoning of their expressive power and some sketch of proofs.


In addition to recursion, another desirable addition to database query languages 
is negation. In this paper, we explore a weaker version of classical negation called 
\emph{negation as failure} to specify nonmonotonic programs. However, a logic 
program with negation as failure might have more than one minimal Herbrand model 
\footnote{A set of ground terms of a program $P$, whose interpretation 
makes every ground term denote itself.}. Hence, it is important 
to come up with a semantics that defines the correct minimal 
model for a program with and without negation. Two such semantics 
are stable models and well-founded models. A conventional 
definition of a stable model
describes a set $M$ of ground atoms that matches with the least model 
of a grounded program whose negative literals are eliminated with respect to 
$M$. In this sense, the stable model semantics coincides with 
the least model semantics on positive programs. Under this 
semantics, a program can also have multiple stable models. Even though 
this problem can be fixed using the well-founded semantics, 
it makes the stable model semantics suitable for solving search 
problems whose solutions are numerous and incomparable. Such 
application is found in answer set programming (ASP), wherein 
solutions to a search problems are computed as stable models 
that satisfy both the program specifications and additional constraints. 
This type of logic programming based on stable models has lower 
expressive power than the conventional Horn logic programming 
\footnote{Logic programs composed of Horn clauses, 
each of which is a disjunction of literals (atoms or negated atoms) with at most one of 
them positive.}. While Horn programs can express any recursively enumerable set, 
ASP programs are only able to solve decision problems in the NP class. 
Nonetheless, the stable model semantics is still 
a powerful computational tool in artificial 
intelligence and operations research where focus 
revolves around many search and constraint satisfaction 
problems in NP.

(Well-founded semantics)

The organization of this paper is as follows. First, we 
explore the extension of the relational calculus with recursion 
using the least fixpoint semantics. We then discuss the effects 
of negation in logic programming by describing the nonmonotonic 
formalism behind negation as failure and the problem of 
having more than one minimal Herbrand models. Section 4 and 5 
study the stable model and well-founded semantics as the two 
commonly accepted mechanisms to represent logic programs with 
negation. The last section puts our work into perspective.
\section{Least Fixpoint Query} \section{Least Fixpoint Query}
The least fixpoint query we discuss here refer to domain relational calculus with least fixpoint operator extension, as described in textbook\cite{Abiteboul1}, which is intuitively least fixpoint logic under the context of relational database theory. We first briefly describe its syntax and sematics in the first section, where we skip the part belong to relation calculus and focus on the definition of least fixpoint operator. Then we will put our attention on the discuss of its expressive power and data complexity in second one. \\
We are also going to cover several important theorems related to these topics and probably contain some simple sketch of proof.

\subsection{Syntax and semantics}
The syntax and semantics of least fixpoint query are mostly the same as domain relational calculus except the added least fixpoint operator. To define the notion of fixpoint operator, we need first introduce the notion of operator. \\
An operator $\Phi: \textit{P}(A^k)\rightarrow \textit{P}(A^k)$ is a mapping defined on the set of all k-ary relations on the universe $A$ of structure $\textbf{A}$, where universe and structure are basic concepts of finite model theory (for detail, check section 2.2 in \cite{kolaitis1}). Under context of domain relational calculus, we could consider the structure as database instance and its universe as the active domain of that database instance. Then if a k-ary relation $P$ satisfy $P=\Phi(P)$, then P is said to be a fixpoint of $\Phi$. And if a fixpoint $P^*$ of $\Phi$ satisfy $P^*\subseteq P$ for all fixpoint $P$ of $\Phi$, then $P^*$ is said to be a least fixpoint of $\Phi$.\\
Intuitively, an operator does not necessary have a fixpoint point. So we need to add some contrains to the operator we are dicussing to ensure the least fixpoint's existence. And these contains are addressed by a theorem called Knaster-Tarski Theorem\cite{Tarski}. \\

Before describing what the Knaster-Tarski Theorem says, we need to define some basic concepts first. For every ordinal $\alpha$, we define 
$$\Phi^{\alpha}=\Phi(\cup_{\beta<\alpha} \Phi^{\beta})$$
where $\Phi^0=\emptyset$. And we say an operator $\Phi$ is monotone if 
$$\Phi\models(\forall P_1,P_2\in P(A^k)(\Phi(P_1)\subseteq\Phi(P_2)))$$
Then the Knaster-Tarski Theorem basically states that if operator $\Phi$ is monotone, then $\Phi$ has a least fixed point $\textbf{lfp}(\Phi)$ and there exits an ordinal $s\leq |A|^k$ (assume active domain A is finite) such that $\textbf{lfp}(\Phi)=\Phi^{\infty}=\Phi^s=\Phi^{\delta}$, for all $\delta > s$.\\
After defined the notion of least fixpoint of operator, we are now going to discuss how to argument relational calculus with it. A natural approach is to express the required monotone operator as relational calculus expression and then use its least fixpoint as an exiension. But M. Ajtai and Y. Gurevich (1987,\cite{Ajtai}) had shown the problem of monotonicity of a first-order formula is undecidable. So instead they introduced a new notion called positivity, which can implies monotonicity but is decidable. The notion of positivity can be stated as follows: a first order logic formula $\varphi(S)$ that cantains a k-ary relation S is said to be positive in S if every occurance of S in $\varphi(S)$ is under an even number of negation. And if a first order logic formula $\varphi(S)$ is positve in S, then it (under context of relational database theory) is monotone on all database instances under the database schema where the formula is defined.\\ 
Now after restrict the relational calculus formula we used to define operator to be positive, each formula is now assciated with a least fixed point. And according to the Knaster-Tarski Theorem, we also it can be computed by repeatively apply the corresponding oeprater. We can now define the syntax and sematics of our least fixpoint query.(for formal definition of least point logic, see section 2.6 in \cite{kolaitis1})\\

\begin{description}

\item[Syntax:]
Let $\varphi(\textbf{x},\textbf{y}, S_1,S_2,..,S_n, T)$ be a relational calculus formula, where $\textbf{x}=(x_1, x_2,...,x_k)$, $\textbf{y}=(y_1,y_2,...,y_m)$ are variables, $S_1,S_2,..,S_n, T$ are relation symbols and $\varphi$ is positive in $T$. Let $\textbf{u}=(u_1, u_2,...,u_k)$ be a k-tuple of variable not occuring in $\varphi$. Then $[\textbf{lfp}\,T\textbf{x}.\varphi](u)$ is a least fixpoint formula. And more specifically, if $\varphi$ only consists of $\textbf{x}$ and $T$, then we can write the corresponding least fixpoint formula compactly as $\mu_T(\varphi)(u)$. We call $\mu_T$ the least fixpoint operator. Farther more, the queries that are definable by least fixpoint formula are called least fixpoint query.\\

\item[Sematics:]
Given database instance $A$ and a tuple $a$, then $A, a \models [\textbf{lfp}\,T\textbf{x}.\varphi](u)$ if $A, a \models R(u)$ where R is interpreted as a relation such that $R=\textbf{lfp}(\Phi)$, where $\Phi$ is a $k$-ary operator such that
$$ \Phi(T) = \{\textbf{x}\in adom(A)^k: A\models \varphi(\textbf{x},\textbf{y},S_1,S_2,..,S_n,T)\}$$ 

\end{description}

In the following section, we only consider the least fixpoint formulas that contain $\mu_T(\varphi)(u)$. In other words, formulas that do not allow parameters to appear inside the input formula of least fixpoint operator. We do it for simplicity to avoid the complexity of nested least fixpoint queries.

\subsection{Expressive Power and Data Complexity}

In this section, we are going to discuss the expressive power of least fixpoint query. As an extension to relational calculus query, least fixpoint query has higher expressive power. We can illustrate it via a P-complete decision problem called \textbf{Path Systems}\cite{Cook}. Path Systems is a decision problem not definable in relational calculus but definable in our least fixpoint query. \\

\begin{description}
\item[Path Systems:]
Let $F$ be a unary relation representing a set of formulas, $A$ be a unary relation representing a set of axioms, and $R$ be a ternary relation representing a set of rules of inference. Then for a database instance $S = (F, A, R)$ , we say a formula $f\in F$ is theorem of $S$ if $f\in A$ or there exist other two theorems $g$ and $h$ of $S$ and a tuple $(f, g, h)$ in $R$. Then the Path System problem can be stated as follows: Given such an database instance S, and a formula $f\in F$, is $f$ a theorem of $S$?
\end{description}

\begin{claim}
Path Systems Problem is definable in least fixpoint query.
\end{claim}

It's very straight forward to show above decision problem is definable in our least fixpoint query, by directly translate its definition into following boolean least fixpoint formula:
$$ \mu_T(A(x)\lor (\exists g \exists h)(T(g)\land T(h) \land R(x, g, h)))(f) $$
And since we reduced a P-complete problem to a specific least fixpoint formula, we also proved that the data complexity of our least fixpoint query is at least P, P-hard in other word.\\

Proving this Path System problem is not definable in relational calculus is relatively difficult. To do that, we need to introduce a useful notion, called Ehrenfeucht-Fra\"iss\'e games, which is commonly used for proving certain problem is undefinable in certain logic. The method of Ehrenfeucht-Fra\"iss\'e games has been proven to be an effective tool in studying the limitation of expressive power of first-order logic and many other logic of higher expressive power.

\begin{description}
\item[Ehrenfeucht-Fra\"iss\'e games]
Given two database instances $A$ and $B$ of the same database schema. A $r$-move Ehrenfeucht-Fra\"iss\'e game on $A$ and $B$ has two players, Spoiler and Duplicator. They play the game in turn. In each turn $1\leq i \leq r$, the Spoiler first picks an element $a_i$ from $adom(A)$ (or $b_i$ from $adom(B)$), then the Duplicator picks an element $b_i$ from $adom(B)$ (or $a_i$ from $adom(A)$ if Spoiler picked from $adom(B)$). After $r$ turns, a picking sequence is generated $(a_1, b_1),...,(a_i,b_i),...,\\(a_r, b_r)$, which is called a run of $r$-move Ehrenfeucht-Fra\"iss\'e game. Then the Duplicator wins this run if the mapping $f$ of form
$$ f: a_i\rightarrow b_i, 1\leq i \leq r$$
is a partial isomorphism between $A$ and $B$, otherwise the Spoiler wins this run. And if the Duplicator has a winning strategy to allow it to win every run, the Duplicator wins the $r$-move Ehrenfeucht-Fra\"iss\'e game on $A$ and $B$, otherwise the Spoiler wins.
\end{description}

The relation between the definability of a decision problem in relational calculus and Ehrenfeucht-Fra\"iss\'e games is addressed by a famous method, which is called \textbf{The Method of Ehrenfeucht-Fra\"iss\'e games} (for formal definition and proof, check \cite{kolaitis1}). We just (informally) describe it under the context of relational database theory.

\begin{description}
\item[The Method of Ehrenfeucht-Fra\"iss\'e games for Relational Calculus]
\end{description}
Let $Q$ be a boolean query defined on database schema $S$, then $Q$ cannot be expressed as a relational calculus expression on $S$ if for every positive integer $r$ there exist two database instances $A_r, B_r$ on $S$ such that
\begin{enumerate}
    \item $Q(A_r)=1\land Q(B_r)=0$.
    \item The winner of $r$-move Ehrenfeucht-Fra\"iss\'e game on $A_r$ and $B_r$ is Duplicator.
\end{enumerate}
Now let us prove the Path System problem cannot be expressed by a relational calculus expression.

\begin{claim}
Path Systems problem is not definable in relational calculus.
\end{claim}

\begin{proof}
Let two database instances $I$ and $J$ of the same schema be the following:
$$I=(\{F(f), F(k)\}, \{A(g),A(h)\},\{R(f,g,h)\})$$
$$J=(\{F(f), F(k)\}, \{A(g),A(h)\}, \{R(k,g,h)\})$$
Then for a query $Q$ asking whether $f$ is a theorem we have $Q(A)=1 \, \text{and} \, Q(B)=0$. Then the winning strategy of the Duplicator is: for every $f,k,g,h$ that Spoiler picks, Duplicator picks $k,f,g,h$ correspondingly. Notice that this strategy makes Duplicator win $r$-move Ehrenfeucht-Fra\"iss\'e games for every positive integer $r$. So we can conclude that query $Q$ is not definable in relational calculus.
\end{proof}

After prove Path System problem is not definable in relational calculus but definable in our least fixpoint query, we therefore proved the least fixpoint operator we introduced is a effective extension to relational calculus.\\

Then, after talk about the improved expressive power of least fixpoint query. Now let's discuss the data complexity. As we have seem before, by reduce Path System problem to least fixpoint query, we saw least fixpoint query's data complexity is P-hard. Now we are going to see it is actually P-complete.\\

\begin{claim}
Data Complexity of least fixpoint query is P-complete.
\end{claim}
\begin{proof}
Recall in the syntax of least fixpoint query, each least fixpoint formula is required to be positive. This property lead to the monontonicity of its corresponding k-ary operator $\Phi$. And also recall the Knaster-Tarski Theorem which states given a database I, we have $\textbf{lfp}(\Phi)=\Phi^s$ where $s<adom(I)^k$. So the number of iterations takes for the operator $\Phi$ to reach least fixed point is in polynormial of size of database. Then since the operator $\Phi$ is defined by ordinary relational calculus formula, and the data complexity of relational calculus is P. So each iteration also takes polynormial. Combine them togeather, we have data complexity of the original least fixpoint formula in P. Finally, as we have seem Path System problem which is P-complete can be reduced to a least fixpoint query , the data complexity of least fixpoint query is also P-complete. 
\end{proof}

After describe the expressive power and data complexity of least fixpoint query, one thing also worth mentioned is: Although least fixpoint query is more powerful than relational calculus and it contains P-complete query like Path System Problem, it still have serious limition. There are still many simple problems undefinable in least fixpoint query. E.g. the Even Cardinality problem\cite{kolaitis1}. From which we see even with its P-complete data complexity there are still many problem in P laying outside of least fixpoint query' expressive power.
\section{Negation in Logic Programming} We have discussed the extension of the relational calculus with recursive queries 
using the least fixpoint semantics. Another desirable feature that is absent from 
basic logic programming is negation. There are two forms of negation: strong (``classical'') 
negation to express falsity and negation as failure to express non-truth. The 
latter is a weaker version of negation: instead of concluding 
$\neg p$ if $p$ is true, we conclude $not \: p$ if $p$ is not known to be true. 
Classical negation can be problematic in logic programming that computes 
recursively enumerable relations because their complements cannot be computed [1]. 
Going forward, we will instead focus on logic programs with negation as failure. 

The theory behind negation as failure arises from nonmonotonic reasoning, which states 
the possibility of retracting a conclusion given new information [5]. For instance, 
Consider the following logic program with negation as failure:
\begin{align}
    \begin{split}
        & p(a). \hspace{1em} p(b). \hspace{1em} q(a). \\
        & r(X) \leftarrow p(X), not \: q(X).
    \end{split}
\end{align}
we can conclude $r(b)$ from 
program (1), but this conclusion is invalid if fact $q(b)$ is added. Logic programming with 
negation can be translated into two nonmonotonic formalisms: default and autoepistemic logic. 
Autoepistemic logic couples each set $A$ of axioms with the operator $L$ to express the idea that 
$A$ is believed to be true, or $LA$. Conversely, the expression $not \: A$ is analogous to the 
autoepistemic formula $\neg LA$. We can rewrite program (1) using autoepistemic logic:
\begin{align}
    \begin{split}
        & p(a). \hspace{1em} p(b). \hspace{1em} q(a). \\
        & r(a) \leftarrow p(a), \neg Lq(a). \\ 
        & r(b) \leftarrow p(b), \neg Lq(b).
    \end{split}
\end{align}
where each ground negative literal $\neg B$ becomes $\neg LB$. It follows from the 
autoepistemic theory that program (2) has a unique stable expansion, whose atoms 
form the intended model. Observe that the program is 
grounded to remove its variables before the translation. Autoepistemic logic provides an intuitive 
representation of negation as failure as the absence of belief. 

A logic program with negation can also be translated into default logic, where the head 
of each rule becomes the conclusion, the conjunction of positive literals becomes the premise, 
and each negative literal becomes a justification $M \neg A$ (which reads as ``it is consistent to assume $\neg A$''). 
Accordingly, program (1) corresponds to the following default logic:
\begin{align}
    \begin{split}
        \frac{}{p(a)} \hspace{1cm} \frac{}{p(b)} \hspace{1cm} &\frac{}{q(a)} \\ 
        \frac{p(X) \: : \: M \neg q(X)}{r(X)}&
    \end{split}
\end{align}
Even though default logic allows the use of variables in a default, each variable is simply 
the syntactic sugar for a set of its ground instances. Thus, programs (2) and (3) are essentially 
the same: asserting $q(a)$ and $q(b)$ are not believed is synonymous to asserting that it is consistent to assume 
$\neg q(X)$. Scoping a logic program within the perimeter of its ground instances is a key step 
towards the first definition of a stable model. 

Adding negation to Horn logic programs, however, may result in the existence of multiple minimal 
Herbrand models. 
Program (1) has two minimal models, $\{p(a),p(b),\\q(a),r(b)\}$ and $\{p(a),p(b),q(a),q(b)\}$. 
Observe that the latter model is considered ``bad'' because $q(b)$ is not supposed to be true. 
The major challenge when extending logic programming with negation is to invent a semantics that 
captures the correct minimal model of a program. 
Two approaches are proposed to preserve the notion of a single intended model for a logic program 
in the presence of negation. 
Apt, Blair, and Walker [6] introduce a stratified class of logic programs in which recursions and negation are 
mutually exclusive. In this case, there exists a single \emph{perfect model} for every stratified program. 
This technique, however, comes at the expense of restricting the semantics of logic programs with negation. 
The second approach removes this restriction by assigning to an arbitrary program a single 3-valued 
model using the well-found semantics. We explain this semantics in more detail in Section 5. 
On the other hand, we also have what is called the stable model semantics, which 
assigns a collection of intended models rather than a single one to a logic program with negation. 
Even though this property violates the notion of a single model that we are trying to salvage when 
extending logic programming with negation, it makes the stable model semantics a useful 
computational technique for solving problems with many solutions. 
The following section discusses the semantics and applications of stable models. 
\section{Stable Model Semantics} Consider the following logic program with negation:
\begin{align}
    \begin{split}
        & p(a). \hspace{1em} p(b). \hspace{1em} q(a). \\
        & r(X) \leftarrow p(X), not \: q(X).
    \end{split}
\end{align}
which has two minimal Herbrand models:
\begin{align*}
    \{p(a),p(b),q(a),r(b)\}
\end{align*}
and 
\begin{align*}
    \{p(a),p(b),q(a),q(b)\}
\end{align*}
Accordingly, adding negation to Horn logic programs may result in the existence of multiple minimal Herbrand models. 
Two approaches are proposed to preserve the notion of a single intended model for a logic program. 
Apt, Blair, and Walker [6] introduce a stratified class of logic programs in which recursions and negation are 
mutually exclusive. In this case, there exists a single \emph{perfect model} for every stratifed program. 
This technique, however, comes at the expense of restricting the semantics of logic programs with negation. 
The second approach removes this restriction by assigning to an arbitrary program a single 3-valued 
model using the \emph{well-found semantics}. We explain this semantics in more detail in Section 8. The 
well-founded semantics is a specification of what is called the stable model semantics, which 
assigns a collection of intended models rather than a single one to a logic program with negation.

In this section, we discuss the semantics and applications of stable models. There is a large body 
of study on the concept of a stable model based on different interpretations. To put our report into 
context, we first cover the classic grounding definition of stable models introduced in Gelfond and Lifschitz [2], 
which provides an intuitive mechanism using autoepistemic logic to model logic programs with negation. 
Then, we describe the modern  
technique of circumscription to obtain a stable model by translating a logic program 
into a formula of second-order logic (4). The second interpretation of stable models gives semantics 
to nontraditional constructs found in answer set programming (ASP). Subsection 7.2 examines the role of 
stable model semantics in logic programming and compares it with traditional Horn logic. Finally, 
we illustrate the application of stable models in ASP. 
 
\subsection{Definition}
The theory behind stable models arises from nonmonotonic reasoning, which states the possibility 
of retracting a conclusion given new information [5]. For instance, we can conclude $r(b)$ from 
program (1), but this conclusion is invalid if fact $q(b)$ is added. Logic programming with 
negation can be translated into two nonmonotinic formalisms: default and autoepistemic logic. 
Autoepistemic logic couples each set $A$ of axioms with the operator $L$ to express the idea that 
$A$ is believed to be true, or $LA$. Conversely, the expression $not \: A$ is analogous to the autoepistemic formula 
$\neg LA$. We can rewrite program (1) using autoepistemic logic:
\begin{align}
    \begin{split}
        & p(a). \hspace{1em} p(b). \hspace{1em} q(a). \\
        & r(a) \leftarrow p(a), \neg Lq(a). \\ 
        & r(b) \leftarrow p(b), \neg Lq(b).
    \end{split}
\end{align}
where each ground negative literal $\neg B$ becomes $\neg LB$. It follows from the 
autoepistemic theory that program (2) has a unique stable expansion, whose atoms 
form the intended model. Observe that the program is 
grounded to remove its variables before the translation. Autoepistemic logic provives an intuitive 
representation of negation as failure as the absence of belief. 

A logic program with negation can also be translated into default logic, where the head 
of each rule becomes the conclusion, the conjunction of positive literals becomes the premise, 
and each negative literal becomes a justification $M \neg A$ (which reads as ``it is consistent to assume $\neg A$''). 
Accordingly, program (1) corresponds to the following default logic:
\begin{align}
    \begin{split}
        \frac{}{p(a)} \hspace{1cm} \frac{}{p(b)} \hspace{1cm} &\frac{}{q(a)} \\ 
        \frac{p(X) \: : \: M \neg q(X)}{r(X)}&
    \end{split}
\end{align}
Even though default logic allows the use of variables in a default, each variable is simply 
the syntactic sugar for a set of its ground instances. Thus, programs (2) and (3) are essentially 
the same: asserting $q(a)$ and $q(b)$ are not believed is synonymous to asserting that it is consistent to assume 
$\neg q(X)$. Scoping a logic program within the perimeter of its ground instances is a key step 
towards the first definition of a stable model. 

\subsubsection{Gelfond-Lifschitz Reduct}
The most fundamental definition of a stable model is based on the Gelfond-Lifschitz transformation 
process [2] of removing negations from a logic program $\Pi$. 
We first ground $\Pi$ so that each variable occurence in $\Pi$ is replaced by its ground instances. 
The reduct of $\Pi$ relative to a set $M$ of its ground atoms is obtained by the following two 
elimination procedures:
\begin{enumerate}[label=(\roman*)]
    \item Eliminating each rule with a negative literal $not \: A_i$ with $A_i \in M$, and 
    \item Eliminating all the negative literals from the bodies of the remaining rules.
\end{enumerate}
The reduct of $\Pi$ follows the intuition behind a stable expansion in 
autoepistemic logic. Here, $M$ is considered the possible set of beliefs that a rational 
agent might hold, given $\Pi$ as his premises [2]. If a rule $R$ contains a literal $not \: A_i$ 
such that $A_i \in M$, then we cannot establish $not \: A_i$. Hence, $R$ is false in $M$ and can be 
removed from $\Pi$, as stated in procedure (i). Otherwise, if $A_i \not \in M$, we can conclude 
$not \: A_i$, and this literal is trivial. Removing $not \: A_i$ from the body of $R$ in 
procedure (ii) does not change the semantics of the program. Since the reduct of $\Pi$ no longer 
contains any negation, it has a unique minimal Herbrand model.  
It follows that $M$ is a stable model of $\Pi$ if it coincides with such minimal Herbrand model.

Let $\Pi$ be program (1), where each rule is replaced by its ground instances 
\begin{align*}
    & p(a). \hspace{1em} p(b). \hspace{1em} q(a). \\
    & r(a) \leftarrow p(a), \neg q(a). \\ 
    & r(b) \leftarrow p(b), \neg q(b).
\end{align*}
Let $M = \{p(a),r(a)\}$. Then the reduct of $\Pi$, $\Pi _M$, is 
\begin{align*}
    & p(a). \hspace{1em} p(b). \hspace{1em} q(a). \\
    & r(a) \leftarrow p(a). \\ 
    & r(b) \leftarrow p(b).
\end{align*}
The minimal Herbrand model of $\Pi _M$ is $\{p(a),p(b),q(a),r(b),r(a)\}$, which does not 
match $M$. Hence, $M$ is not a stable model. Let us pick a different 
$M = \{p(a),p(b),q(a),\\r(b)\}$. Then $\Pi _M$ is
\begin{align*}
    & p(a). \hspace{1em} p(b). \hspace{1em} q(a). \\
    & r(b) \leftarrow p(b). 
\end{align*}
In this case, the least model of $\Pi _M$ coincides with $M$, and $M$ is stable.

The Gelfond-Lifschitz reduct provides an intuitive way to represent a logic 
program with negation by removing the negation itself. However, this definition 
of a stable model has its limitations. First of all, the reduction process relies 
on the grounding of a logic program, which can prove impractical in the presence of 
function symbols. In this case, there are infinitely many ground instances. 
Secondly, the reduct of a logic program serves only to verify whether a predetermined 
set $M$ of ground atoms is stable or not. It is more convenient to have a systematic 
technique to generate a stable model without guessing. The next definition of a stable 
model helps offset these limitations.

\subsubsection{Circumscription}
Similar to autoepistemic and default logic, circumscription is another variant of 
nonmonotonic reasoning that exerts minimization on the extension of predicates in 
second-order logic [5]. We have been using minimal Herbrand models to describe the semantics 
of logic programs. Yet, a logic program can also be seen as a representation of a set of 
first-order formulas called \emph{program completion}. For instance, program (1) 
represents the following formula
\begin{align}
    \forall x (p(x) \leftrightarrow x = a \lor x = b) \land 
    \forall x (q(x) \leftrightarrow x = a) \land 
    \forall x (p(x) \land \neg q(x) \rightarrow r(x))
\end{align}
Given a program completion $F$, we first replace all the occurences of negative literals 
with a new symbol $p'$. The circumscription of $F$, $CIRC[F]$, is denoted as 
\begin{align}
    F \land \neg \exists u((u < p) \land F(u)) \land \forall x (p'(x) \leftrightarrow p(x))
\end{align}
where $p$ is the set of all predicate constants in $F$, $u$ is the set of distinct 
predicate variables of the same length, and $F(u)$ substitutes the variables $u$ for 
the constants $p$. The second conjunctive term of $CIRC[F]$ expresses the minimal 
condition of the predicates $p$ in the sense that we cannot further reduce $p$ without 
invalidating $F$. The third conjunctive term asserts that each new predicate is 
equivalent to its old predicate.

For example, let $F$ be formula (4) where each negative literal is renamed.
\begin{align*}
    \forall x (p(x) \leftrightarrow x = a \lor x = b) \land 
    \forall x (q(x) \leftrightarrow x = a) \land 
    \forall x (p(x) \land \neg q'(x) \rightarrow r(x))
\end{align*}
then $CIRC[F]$ is 
\begin{multline*}
    \forall x (p(x) \leftrightarrow x = a \lor x = b) \land 
    \forall x (q(x) \leftrightarrow x = a) \land 
    \forall x (p(x) \land \neg q'(x) \rightarrow r(x)) \\ 
    \land \neg \exists uvz (
        \forall x (u(x) \leftrightarrow x = a \lor x = b) \land 
        \forall x (v(x) \leftrightarrow x = a) \land 
        \\ \forall x (u(x) \land \neg q'(x) \rightarrow z(x)) 
    ) 
    \land \forall x (q'(x) \leftrightarrow q(x))
\end{multline*}

\subsubsection{Modified Circumscription}
One advantage that circumscription has over the Gelfond-Lifschitz reduct 
is the omission of grounding as the preliminary step to determine a stable 
model. Ferraris et al. [4] proposes a modification of circumscription of 
a program completion $F$, $SM[F]$, which is denoted as 
\begin{align}
    F \land \neg \exists u((u < p) \land F^*(u)) 
\end{align}
where $u$, $p$ are the same symbols used for the original circumscription formula (5), 
and $F^*(u)$ 
replaces each predicate constant $p_i$ that is not a negation with 
the corresponding predicate variable $u_i$. Hence, formula (4) could 
have been simply rewritten as 
\begin{multline*}
    \forall x (p(x) \leftrightarrow x = a \lor x = b) \land 
    \forall x (q(x) \leftrightarrow x = a) \land 
    \forall x (p(x) \land \neg q'(x) \rightarrow r(x)) \\ 
    \land \neg \exists uvz (
        \forall x (u(x) \leftrightarrow x = a \lor x = b) \land 
        \forall x (v(x) \leftrightarrow x = a) \land 
        \\ \forall x (u(x) \land \neg q(x) \rightarrow z(x)) 
    ) 
\end{multline*}

It follows 
that any Herbrand interpretation that satisfies $SM[F]$ is a stable model. 
Note that $SM[F]$ is identical to $CIRC[F]$ if $F$ is negation-free. 
$SM[F]$ differs from $CIRC[F]$ by not replacing  
negated predicates with auxiliary predicates and adding an additional 
conjunctive term to relate the two. In addition to the omission of 
grounding, the modified version of circumscription 
extends the stable model semantics to programs with nontraditional constructs 
(such as choice rules, disjunctive rules, and constraints) used in ASP as 
long as they are expressible in first-order logic. The Gelfond-Lifschitz reduct, 
on the other hand, is only applicable to programs in the form of Horn clauses.

\subsection{Properties}
Because of its intuitive definition, the stable model semantics is commonly 
accepted as the standard for logic programming with negation. The 
following are the properties of the stable model semantics:
\begin{enumerate}[label=(\roman*)]
    \item Any stable model of a logic program $\Pi$ is its minimal Herbrand model [2]. 
    \item A well-founded model of $\Pi$ defines its unique stable model [2]. 
    \item If $\Pi$ is stratified, then there is a unique stable model that coincides 
    with the perfect model of $\Pi$ [2].
    \item Any stable model of $\Pi$ is a subset of the set of head atoms in $\Pi$ [8].
    \item $\Pi$ might have several stable models, or no stable models at all. For 
    instance, the program consisting of one rule $p \leftarrow not \: p$ has no 
    stable models, while the program with two rules $p \leftarrow not \: q$ 
    and $q \leftarrow not \: p$ has two stable models, $\{p\}$ nad $\{q\}$. The 
    well-founded semantics helps solve this shortcoming of the stable model 
    semantics by ensuring the existence of a unique stable model for $\Pi$ [1].
    \item If there are several stable models for $\Pi$, they are both minimal 
    and incomparable [1]. 
    \item The problem of determining the existence of a stable model for a finite 
    propositional logic program is NP-complete [6]. This follows from the expressive 
    power of the stable model semantics to represent all decision and search 
    problems in NP. 
\end{enumerate}

\subsection{Stable Logic Programming} 
We have seen that a logic program might have several incomparable stable models, 
which brings into question the use of stable models as the semantics for 
logic programming with negation. Rather 
than trying to reconcile the stable model semantics with Horn logic programming, 
Marek and Truszczynski [6] argue that the lack of a single intended model 
makes the stable model semantics suitable for representing 
some constraint satisfaction problems whose solutions 
are a finite family of finite sets to [6]. 
They refer logic programming based on the stable model semantics  
as stable logic programming (SLP). A decision problem that is associated with 
stable models is whether a finite logic program has a stable model. If we 
encode an instance $I$ of a decision problem $\Pi$ in the class NP as a propositional program 
$P ^I _ \Pi$, then $\Pi$ has a solution for $I$ if and only if $P ^I _ \Pi$ 
has a stable model [6]. Hence, the problem of the existence of a stable model is 
NP-complete, as stated in property (vii). In comparison to Horn logic programming 
which can specify any recursively enumerable set, SLP is only able 
to solve NP problems [6], which, however, are among the most important computational 
problems in computer science.

Unlike Horn logic programming, SLP disallows the use of function symbols in favor of 
a finite set of grounding instances and stable models. The syntax of SLP emulates 
that of DATALOG with negation. The absence of function symbols, however, restricts the 
use of recursion in SLP. In fact, recursion is defined in terms of intensional predicates, 
and not in terms of function symbols additionally used in Horn logic programming. Instead 
of constructing stable models recursively, SLP first specifies a set of potential candidate 
stable models, and then eliminates those that do not satisfy certain constraints. This 
generate-and-test methodology is the basis of answer set programming. 

One key feature of logic programming is the separation of logic from control. The logic 
of a program itself determines the solutions without any execution specifications from 
the programmer. In Horn logic programming, the control of a program is accomplished 
using the \textit{SLD-resolution} to compute the solutions [6]. Accordingly, SLP also 
has some uniform control mechanisms to process SLP programs and compute their stable models. 
Since there are no function symbols in SLP, it is possible to exhaust all subsets of the 
finite Herbrand base of a SLP program for stable model candidates. Both the 
expressive power to encode NP search problems and the availability of 
a uniform control mechanism make SLP a useful computational tool. One such example is answer set 
programming.

\subsection{Answer Set Programming}
Answer set programming is a form of stable logic programming that solves difficult 
search problems [8]. ASP uses an answer set solver such as \textit{smodels} as a uniform control 
to compute stable models as solutions to a search problem. The frontend of \textit{smodels}, 
called \textit{lparse}, defines a set of constructs to formulate a search problem before 
being analyzed by the answer set solver. 

\subsubsection{Syntax}
Programs 
with multiple stable models usually consist of choice formulas. For example, the rule 
\begin{align}
    {q(X)} \: \mathop{:\!\!-} \: p(X).
\end{align}
describes all possible subsets of a given set $p$. 

\textit{lparse} also provides cardinality expressions $l \: \{...\} \: u$ to impose numerical 
bounds on a choice rule, where $l$ is a lower bound and $u$ is an upper bound. We can change 
rule (7) to contain a lower bound for the head atom, 
\begin{align*}
    1 \: \{q(X)\} \: \mathop{:\!\!-} \: p(X).
\end{align*}
to consider only subsets of $p$ with at least one element. 

Another useful ASP construct is the use of disjunction in the head of a rule. The 
rule 
\begin{align*}
    q(X) \: ; \: r(X) \: \mathop{:\!\!-} \: p(X).
\end{align*}
enumerates all possible partitions of a set $p$ into two disjoint subsets $q$ and $r$. 

A conditional literal $\{l:l_1:...:l_n\}$, where $l_i$ are literals, describes a set of 
literals $l$ if $l_1,...,l_n$ are true. For instance, instead of listing all possible 
ground instances of $r(X)$ as $\{r(a),r(b),r(c)\}$, we can take advantage of a conditional 
literal as follows:
\begin{align*}
    &p(a). \hspace{1em} p(b). \hspace{1em} p(c). \\
    &\{r(X) \: : \: p(X)\}.
\end{align*}

So far, we have only discussed programs with negation as failure. ASP also supports 
strong negation in the form of $-p$ to express ``$p$ is false'' as opposed to 
the negation as failure $not \: p$ to express ``$p$ is not known to be true''. Combining 
both kinds of negation $-p \: \mathop{:\!\!-} \: not \: p$ gives the closed world assumption 
stating that ``$p$ is false if there is no evidence that it is true'' [8]. Strong negation 
is useful to represent negative facts in an incomplete state of knowledge.

Finally, a constraint is a rule with an empty head 
\begin{align*}
    \mathop{:\!\!-} \: p
\end{align*}
which prohibits 
any generation of a literal $p$. Constraints appear in almost every ASP program to weed out 
``bad'' stable models. All of the constructs in ASP are expressible in first-order 
logic [3]. Hence, the stable models of an ASP program can be determined using the 
technique of modified circumscription.

\subsubsection{Generate-and-Test}
The logic of an ASP program follows the generate-and-test methodology introduced in 
stable logic programming. The two steps are first (1) to generate all possible stable models 
for a program, and then (2) to select only those that satisfy additional constraints. Let us 
consider an example ASP program to find Hamiltonian cycle in a directed graph [7]. First of all, 
we define a directed graph $G$ with a set of vertices $V$ and a set of edges $E$: 
$vertex(a)$ for all vertices $a \in V$, and $edge(a,b)$ for all directed edges $(a,b) \in E$. Next, 
we generate all potential stable models for this problem using the choice rule 
\begin{align*}
    & \{ \: in(X,Y) \: \} \: \mathop{:\!\!-} \: edge(X,Y). 
\end{align*}
These stable models represent all subsets of edges that could potentially form a Hamiltonian 
cycle. Adding the following constraints 
\begin{align*}
    & \mathop{:\!\!-} \: 2 \: \{in(X,Y) \: : \: edge(X,Y)\}, \: vertex(X). \\
    & \mathop{:\!\!-} \: 2 \: \{in(X,Y) \: : \: edge(X,Y)\}, \: vertex(Y). 
\end{align*}
eliminates subsets of edges that contains pairs of edges starting or ending at the same 
vertex. We further refine the stable models to the problem with these additional clauses: 
\begin{align*}
    & r(X) \: \mathop{:\!\!-} \: in(0,Y), \: vertex(X).\\
    & r(Y) \: \mathop{:\!\!-} \: r(X), \: in(X,Y), \: edge(X,Y). \\ 
    & \mathop{:\!\!-} \: not \: r(X), \: vertex(X).
\end{align*}
Accordingly, each of the final stable models encodes a Hamiltonian cycle wherein every vertex in the graph 
is reachable from an initial 
vertex by a non-empty sequence of $in$ edges.  \\ 

Footnotes maybe?: 
\begin{itemize}
    \item Herbrand models (2 p2): a set of ground terms of a program P, whose interpretation 
    makes every ground term denote itself.
    \item Horn clauses (1 p27): a disjunction of literals (atoms or negated atoms) with at most one of 
    them positive.
    \item Ground atoms: atoms with no variables.
    \item Stratified programs (recusion and negation do not mix) and iterated fixed point model 
    \item Locally stratified programs and perfect models.
\end{itemize}
\section{Well-Founded Semantics} %\begin{align*}, multiline*
Allowing negation in the body of a rule of to have general logic programs may have another problem: As indicated by some research, some programs do not have a satisfactory \emph{total} model. It is thus natural to seek an appropriate \emph{partial} model for such programs.

In this section, \emph{well-founded semantics}, a different approach than stable model semantics in the category of canonical model semantics introduced in [\ref{VanGelder}], is explained and compared to it as well as to other semantics.

\subsection{Preliminaries}
In this subsection, We explain some notions that are used in well-founded semantics.

Given a program $\mathrm{P}$, its \emph{Herbrand universe} is the set of ground terms formed from constants and function symbols in $\mathrm{P}$, and its \emph{Herbrand base} is the set of predicates applied to the ground terms in the Herbrand universe. Note that the program $\mathrm{P}$ may not contain constants or function symbols. In the absence of the former, a constant is added arbitrarily; in the absence of the latter, we then have that both the Herbrand universe and the Herbrand base are countably infinite (otherwise they are finite). An \emph{instantiated rule} of $\mathrm{P}$ is one in which all occurrences of variables are replaced by ground terms in the Herbrand universe. The \emph{Herbrand instantiation} of $\mathrm{P}$ is the set of all instantiated rules.

For a ground atomic formula $p$ (a single predicate applied to ground terms), we say $p$ itself its \emph{positive literal}, and $\neg p$ its \emph{negative literal}; these two literals are \emph{complements} of each other. If $S$ is a set of literals, $\neg \cdot S$ denotes the set obtained by taking the complement of each literal in $S$; we say a literal $q$ is \emph{inconsistent} with $S$ if $q \in \neg \cdot S$; two sets $R$ and $S$ of literals are \emph{inconsistent} if their intersection is nonempty, i.e.\ $R \cap \neg \cdot S \neq \emptyset$; $S$ is \emph{inconsistent} if it is inconsistent with itself, i.e.\ some literal and its complement both appear in $S$; $S$ is \emph{consistent} if it is not inconsistent.

Given a program $\mathrm{P}$, a \emph{partial interpretation} $I$ is a consistent set of literals that are formed from the atomic formulas in the Herbrand base of $\mathrm{P}$. A partial interpretation $I$ is said to be \emph{total} if for any ground atomic formula $p$ in the Herbrand base either of its literal is contained in $I$. A literal $p$ is \emph{true in $I$} if $p \in I$ and is \emph{false in $I$} if its complement is in $I$. This notion can be straightforwardly extended to conjunctions in the normal way.

An instantiated rule of a program $\mathrm{P}$ is \emph{satisfied} in a (partial or total) interpretation $I$ if the head (which is a ground atomic formula) is true in $I$ or some subgoal in the body of that rule is false in $I$; the rule is \emph{falsified} in $I$ if the head is false and all subgoals are true. Moreover, if the head of the rule is false in $I$, but no subgoal is false in it then the rule is said to be \emph{weakly falsified} in $I$. Thus, a rule is weakly falsified if it is falsified, but not vice versa.

A \emph{total model} of a program $\mathrm{P}$ is a total interpretation that satisfies all instantiated rules of $\mathrm{P}$. A \emph{partial model} of $\mathrm{P}$ is a partial interpretation that can be extended to a total model of $\mathrm{P}$.
\ \medskip\\
\noindent\textbf{Remarks.}
\begin{enumerate}[label=(\alph*)]
%
\item The reader may think of a partial interpretation as containing incomplete information: The positive (or negative) literals in it are considered to be true (or false, respectively) atomic facts, and the truth values of the rest of the ground atomic formulas are considered unknown, or unspecified, at least ``at present.''
%
\item Since interpretations are sets, it is natural to consider one to be contained in another if the former is a subset of the latter.
%
\item Note that a partial model $I$ of a program $\mathrm{P}$ may not satisfy some instantiated rules of it, but $I$ can be extended -- adding (possibly no) literals to it -- so that all rules are satisfied. Of course, this cannot be done if $I$ falsifies every instantiated rule; if $I$ only weakly falsifies some rules, then there is still hope for this extension; and \emph{if $I$ does not weakly falsify any instantiated rule, then $I$ is a partial model of $\mathrm{P}$}.
%
\end{enumerate}

\subsection{Unfounded Sets and Well-Founded Partial Models}
Having described the preliminaries in the previous subsection, we explain the notions of unfounded sets and well-founded partial models, which are the main part of [\ref{VanGelder}].

\subsubsection{Unfounded Sets}
Given a program $\mathrm{P}$, let $H$ denote the Herbrand base of $\mathrm{P}$, and assume $I$ is a partial interpretation of $\mathrm{P}$. We say $A \subseteq H$ is an \emph{unfounded set} of $\mathrm{P}$ with respect to $I$ if this condition is true for each ground atomic formula $p \in A$: Either there is no instantiated rule of $\mathrm{P}$ for $p$ or, for each instantiated rule $r$ of $\mathrm{P}$ whose head is $p$, at least one of the following statement is true
\begin{enumerate}[label=(\arabic*)]
%
\item Some (positive or negative) subgoal of the body is false in $I$,
%
\item Some positive subgoal of the body occurs in A.
%
\end{enumerate}
A literal that makes (1) or (2) above true is called a \emph{witness of unusability} for rule $r$ (with respect to $I$).
\medskip\\
\textbf{Remark.} The reader may regard $I$ as a collection of known facts about $\mathrm{P}$ thus far. Rules for which (1) holds cannot be used to derive more facts since its body is false. Rules for which (2) holds have unfoundedness in them: To derive a fact from an unfounded set $A$, i.e.\ to decide the truth value of an atomic formula in $A$, the truth values of some (other) atomic formulas must be decided first.

An example to illustrate the above definition is the following program
\[
\begin{array}{lll}
p(a) & \leftarrow & p(c), not \ p(b) \cr
p(b) & \leftarrow & not \ p(a) \cr
p(e) & \leftarrow & not \ p(d) \cr
p(c) & \ & \ \cr
p(d) & \leftarrow & q(a), not \ q(b) \cr
p(d) & \leftarrow & q(b), not \ q(c) \cr
q(a) & \leftarrow & p(d) \cr
q(b) & \leftarrow & q(a)
\end{array}
\]
The set $\{p(d), q(a), q(b), q(c)\}$ is unfounded with respect to $\emptyset$: There is no instantiated rule for $q(c)$ to decide its truth value; for each of the remaining atomic formulas, condition (2) applies. In contrast, the set $\{p(a), p(b)\}$ is not unfounded because the interdependency between $p(a)$ and $p(b)$ is through negation; as soon as either of them is declared false, the other must be true.

The reader should note that unfounded sets are closed under union. Thus, it is valid to speak of the \emph{greatest unfounded set} of $\mathrm{P}$ with respect to $I$, denoted $\mathbf{U}_\mathrm{P}(I)$, which is defined to be the union of all sets that are unfounded with respect to $I$.
\medskip\\
\textbf{Remark.}
\begin{enumerate}[label=(\alph*)]
%
\item Let $\mathrm{P}$ be a program and $R$ be a set of literals, and let $A$ be an unfounded set of $\mathrm{P}$ with respect to $R$, then for any subset $S \subseteq A$, $A - S$ is unfounded with respect to $R \cup \neg \cdot S$. In other words, if we deduce that certain facts $S$ are in an unfounded set $A$ and add their complements to $R$, the other unfounded atomic formulas remain unfounded.
%
\item Let $I$ be a partial interpretation consisting of a set of positive literals $Q$ and a set of negative literals $\neg \cdot S$. If $I$ does not weakly falsify any instantiated rule of program $P$, then $S$ is an unfounded set with respect to $Q$.
%
\end{enumerate}

\subsubsection{Well-Founded Partial Semantics}
In this section we introduce a (possibly transfinite) sequence of \emph{transformations} on sets of literals in which its limit defines the well-founded semantics. Recall that a transformation $\mathbf{T}$ is \emph{monotonic} if $\mathbf{T}(I) \subseteq \mathbf{T}(I')$ whenever $I \subseteq I'$.

Given a program $\mathrm{P}$ and an interpretation $I$, the three transformations $\mathbf{T}_\mathrm{P}, \mathbf{U}_\mathrm{P}, \mathbf{W}_\mathrm{P}$ are defined as
\begin{enumerate}[label=(\arabic*)]
%
\item $\mathbf{T}_\mathrm{P}(I)$ consists of all positive literals $p$ such that there is an instantiated rule $r$ of $\mathrm{P}$ where $p$ is the head and each (positive or negative) subgoal literal in the body of $r$ is true in $I$.
%
\item $\mathbf{U}_\mathrm{P}(I)$ is the greatest unfounded set of $\mathrm{P}$ with respect to $I$, as introduced in the previous section.
%
\item $\mathbf{W}_\mathrm{P}(I) = \mathbf{T}_\mathrm{P}(I) \cup \neg \cdot \mathbf{U}_\mathrm{P}(I)$.
%
\end{enumerate}
Let a program $\mathrm{P}$ be given and let $H$ denote its Herbrand base. Also, let $\alpha$ range over all countable ordinals. Then the sets $I_\alpha \subseteq H$ are defined recursively by
\begin{enumerate}[label=(\arabic*)]
%
\item $I_0 := \emptyset$.
%
\item For limit ordinal $\alpha > 0$, $I_\alpha := \bigcup_{\beta < \alpha} I_\beta$.
%
\item For successor ordinal $\alpha = \beta + 1$, $I_\alpha := \mathbf{W}_\mathrm{P}(I_\beta)$.
%
\end{enumerate}
Moreover, denote $I^\infty := \bigcup_{\alpha} I_\alpha$. Note that $I^\infty \subseteq H$ also.
\medskip\\
\textbf{Remarks.}
\begin{enumerate}[label=(\alph*)]
%
\item Note that by definition, if a literal $p$ is in $I^\infty$ then the smallest ordinal $\alpha$ for which $p$ appears in $I_\alpha$ is a successor ordinal.
%
\item The sequence of $I_\alpha$'s as defined above is a monotonic sequence of partial interpretations, i.e.\ they are consistent sets of literals.
%
\item By classical results of Tarski, $I^\infty$ is the least fixed point of the transformation $\mathbf{W}_\mathrm{P}$. Since $H$ is countable, $I^\infty = I_\alpha$ for some countable ordinal $\alpha$. The smallest ordinal $\alpha$ for which $I^\infty = I_\alpha$ in the sequence of $I_\alpha$'s is the \emph{closure ordinal} for that sequence. Recall that if $\mathrm{P}$ contains no function symbols then $H$ is finite, and hence the closure ordinal must be finite.
%
\end{enumerate}
We then let $I^\infty$, the fixed point of $\mathbf{W}_\mathrm{P}$, to represent the \emph{well-founded semantics} of program $\mathrm{P}$, so that every positive (or negative) literal in $I^\infty$ denotes that its atomic formula is true (or false, respectively) in $\mathrm{P}$, and that atomic formulas that are missing in $I^\infty$ have no truth values or, ``unknown.''
\medskip\\
\textbf{Remarks.}
\begin{enumerate}[label=(\alph*)]
%
\item \emph{Each $I_\alpha$ as defined above does not weakly falsify any instantiated rule of $\mathrm{P}$ and hence is a partial model of $\mathrm{P}$.}
%
\item We then say that $I^\infty$ is the \emph{well-founded partial model} and, if $I^\infty$ is a total interpretation, i.e.\ if for every $p \in H$ either $p$ or $\neg p$ is in $I^\infty$, then we say $I^\infty$ is a \emph{well-founded model}.
%
\item Thus, every Horn program has a well-founded model $I^\infty$, which is the minimum model in the sense of Van Emden and Kowalski [\ref{VanEmden}], that is, its positive Iiterals are contained in every Herbrand model.
%
\end{enumerate}
\subsection{Comparisons}
Having introduced the well-founded semantics, we discuss below some comparisons between it and other approaches.

Clark introduced the completed program as a way of formalizing the notion that facts not inferable from the rules in the program were to be regarded as false [\ref{Clark}]. The idea is to collect all rules having the same head predicate into a single rule whose body is a disjunction of conjunctions, then replace the symbol $\leftarrow$ (if) by $\leftrightarrow$ (if and only).

The original ``logical consequence'' approach essentially declares that only conclusions that are logical consequences (in the classical, 2-valued sense) of the completed program should be inferred. When the completed program is consistent, this approach implicitly defines a 3-valued interpretation: Assign value true to instantiated atoms that are true in all (2-valued, not necessarily Herbrand) models of the completed program, false to instantiated atoms that are false in all models, and $\bot$ (unknown) to all other instantiated atoms. By [\ref{fitting2}], the completion of every program has a (unique) minimum 3-valued Herbrand model. Fitting suggests that this model be taken for the semantics of the program, which we call the \emph{Fitting model}. 

Having introduced several different approaches, we make the following comparisons between well-founded semantics and the other semantics:
\medskip\\
\textbf{Remark.}
\begin{enumerate}[label=(\alph*)]
%
\item The Fitting model is a subset of $I^\infty$.
%
\item If program $\mathrm{P}$ has a well-founded total model, then that model is the unique stable model.
%
\item The well-founded partial model of $\mathrm{P}$ is a subset of every stable model of $\mathrm{P}$.
%
\item If program $\mathrm{P}$ is locally stratified then it has a well-founded model, which is identical to the \emph{perfect model}.\footnote{See [\ref{prz}] for definition of perfect models.}
%
\end{enumerate}
\section{Conclusion} In conclusion, we have studied the extension of standard query languages with recursion and negation. First, we showed how to embody recursion in relational calculus using the least fixed point, which lead to Least Fixpoint Query. Least Fixpoint Query can express query that are not expressible in relational calculus, e.g.  Path Systems. And we see it makes least fixpoint query having data complexity in P-Complete instead of LOGSPACE in which of relational calculus. And we see important theorem used for analysis of expressive power of relational calculus and logic programs, the Method of Ehrenfeucht-Fra\"{i}ss\'{e} games. From which we can reason Path System is not expressible in relational calculus. Together we see both the expressive power and data complexity of our Least Fixpoint Query. 

Besides recursion, we also explore the effect of adding negation 
to logic programming. In the presence of the negation operator, 
some programs can have more than one least Herbrand model. Therefore, 
we need a different semantics to represent programs with negation 
while still complying with the least model theory. One such 
example is the stable model semantics. The motivation behind 
a stable model is based on the nonmonotonic reasoning, more 
specifically the notion of a unique stable expansion in 
autoepistemic logic. In this sense, negation is represented as 
the absence of belief instead of explicit falsity, thus allowing 
changes in conclusions given new information (or beliefs). The 
most conventional definition of a stable model is described 
using the Gelfond-Lifschitz transformation, which is loosely based on 
autoepistemic logic. Accordingly, a set $M$ of ground atoms is considered 
stable if it matches the least model of the grounded program $\Pi$ reduced 
with respect to $M$. We also studied a more modern definition of a stable 
model using the modified circumscription introduced in Ferraris et al. \cite{lee}. 
This definition removes the necessity of grounding and instead transforms 
a logic program $F$ into a second-order formula, $SM[F]$, that minimizes the extension of 
its predicates. A model of $F$ is stable if it satisfies $SM[F]$. 
Under the stable model semantics, a program is assigned with a 
collection of intended models instead of one. This property 
gives rise to a new paradigm of answer set programming, 
which finds solutions to a search problem as stable models. Applications 
of ASP have been observed in automated product configuration, decision 
support for the space shuttle, and inference of phylogenetic trees.


\begin{thebibliography}{16}
  \bibitem{Abiteboul1}
  Serge Abiteboul, Richard Hull, Victor Vianu.
  Calculus + Fixpoint.
  In \textit{Foundations of Databases}, pages 347-355.
  Addison-Wesley, 1995.

  \bibitem{Abiteboul2}
  Serge Abiteboul, Richard Hull, Victor Vianu.
  First Order, Fixpoint, and While.
  In \textit{Foundations of Databases}, pages 429-459.
  Addison-Wesley, 1995.

  \bibitem{Ajtai}
  M. Ajtai and Y. Gurevich. 
  Monotone versus positive. 
  In \textit{Journal of the ACM}, 34:1004–1015, 1987.

  \bibitem{Cook}
  S. A. Cook. 
  An observation of time–storage trade-off. 
  In \textit{Journal of Computer and System Sciences}, 9:308–316, 1974.

  \bibitem{VanEmden}
  M. H. Van Emden and R. A. Kowalski.
  The Semantics of Predicate Logic as a Programming Language.
  In \textit{Journal of the ACM}, 4:733-742, 1976.

  \bibitem{lee} 
  Paolo Ferraris, Joohyung Lee, and Vladimir Lifschitz. 
  A new perspective on stable models.  
  In \textit{Proceedings of International Joint Conference on Artificial Intelligence (IJCAI)}, pages 372-379. 
  2007.

  \bibitem{ferraris} 
  Paolo Ferraris and Vladimir Lifschitz. 
  Mathematical foundations of answer set programming.  
  In \textit{We Will Show Them! Essays in Honour of Dov Gabbay}, pages 615-664. 
  King's College Publications, 2005.

  \bibitem{fitting}
  Melvin Fitting. 
  Fixpoint semantics for logic programming a survey. 
  In \textit{Theoretical Computer Science}, pages 25-51. Elsevier Science, 2002.

  \bibitem{VanGelder}
  A. Van Gelder, K. A. Ross and J. S. Schlipf.
  The Well-Founded Semantics for General Logic Programs.
  In \textit{Journal of the ACM}, 38:619-649, 1991.

  \bibitem{gelfond} 
  Michael Gelfond and Vladimir Lifschitz. 
  The stable model semantics for logic programming.  
  In \textit{Proceedings of International Logic Programming Conference and Symposium}, pages 1070-1080. 
  MIT Press, 2008.

  \bibitem{kolaitis1}
  Phokion G. Kolaitis.
  On the Expressive Power of Logics on Finite Models.
  In \textit{Finite Model Theory and Its Applications}, pages 27-119
  Springer Verlag, 2007.

  \bibitem{lifschitz1} 
  Vladimir Lifschitz. 
  Datalog Programs and their stable models.
  In \textit{Proceedings of the AAAI Conference on Artificial Intelligence}, pages 1594-1597. 
  MIT Press, 1988.

  \bibitem{lifschitz2} 
  Vladimir Lifschitz. 
  Thirteen definitions of a stable model.  
  In \textit{Lecture Notes in Computer Science}, vol 6300. 
  Springer Verlag, 2010.

  \bibitem{lifschitz0} 
  Vladimir Lifschitz. 
  What is answer set programming? 
  In \textit{Proceedings of the AAAI Conference on Artificial Intelligence}, pages 1594-1597. 
  MIT Press, 1988.

  \bibitem{marek} 
  Victor Marek and Miroslaw Truszczynski. 
  Stable models and an alternative logic programming paradigm.  
  In \textit{The Logic Programming Paradigm: a 25-Year Perspective}, pages 375-398. 
  Springer Verlag, 1999.

  \bibitem{Tarski}
  A. Tarski. 
  A lattice theoretical fixpoint theorem and its applications. 
  In \textit{Pacific Journal of Mathematics}, 5:285–309, 1955.
\end{thebibliography}
\end{document}