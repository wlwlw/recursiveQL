In conclusion, we have studied the extension of standard query languages with recursion and negation. First, we showed how to embody recursion in relational calculus using the least fixed point, which lead to Least Fixpoint Query. Least Fixpoint Query can express query that are not expressible in relational calculus, e.g.  Path Systems. And we see it makes least fixpoint query having data complexity in P-Complete instead of LOGSPACE in which of relational calculus. And we see important theorem used for analysis of expressive power of relational calculus and logic programs, the Method of Ehrenfeucht-Fra\"ıss\'e games. From which we can reason Path System is not expressible in relational calculus. Together we see both the expressive power and data complexity of our Least Fixpoint Query. 