In conclusion, we have studied the extension of standard query languages with recursion and negation. First, we showed how to embody recursion in relational calculus using the least fixed point, which leads to Least Fixpoint Query. Least Fixpoint Query can express queries that are not expressible in relational calculus, e.g.  Path Systems. And we learned that Least Fixpoint Query has data complexity in P-complete instead of LOGSPACE as found in relational calculus. We also covered some important theorems used for the analysis of the expressive power of relational calculus and logic programs, such as the Method of Ehrenfeucht-Fra\"ıss\'e games. We showed that Path Systems is not expressible in relational calculus. Together we deduced both the expressive power and data complexity of our Least Fixpoint Query. 

Besides recursion, we also explored the effect of adding negation 
to logic programming. In the presence of the negation operator, 
some programs can have more than one least Herbrand model. Therefore, 
we need a different semantics to represent programs with negation 
while still complying with the least model theory. One such 
example is the stable model semantics. The motivation behind 
a stable model is based on the nonmonotonic reasoning, more 
specifically the notion of a unique stable expansion in 
autoepistemic logic. In this sense, negation is represented as 
the absence of belief instead of explicit falsity, thus allowing 
changes in conclusions given new information (or beliefs). The 
most conventional definition of a stable model is described 
using the Gelfond-Lifschitz transformation, which is loosely based on 
autoepistemic logic. Accordingly, a set $M$ of ground atoms is considered 
stable if it matches the least model of the grounded program $\Pi$ reduced 
with respect to $M$. We also studied a more modern definition of a stable 
model using the modified circumscription introduced in Ferraris et al. \cite{lee}. 
This definition removes the necessity of grounding and instead transforms 
a logic program $F$ into a second-order formula, $SM[F]$, that minimizes the extension of 
its predicates. A model of $F$ is stable if it satisfies $SM[F]$. 
Under the stable model semantics, a program is assigned with a 
collection of intended models instead of one. This property 
gives rise to a new paradigm of answer set programming, 
which finds solutions to a search problem as stable models. Applications 
of ASP can be found in automated product configuration, decision 
support for the space shuttle, and inference for phylogenetic trees \cite{lifschitz0}.
